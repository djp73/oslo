\section{$N$-adic Integers}
\par The notation used in the definition of the $N$-adic integers will follow
the same notation used by Borevich and Shafarevich in Chapter 1 of
{\em Number Theory}.

\subsection{The $N$-adic Integer Ring}
\par The commutativity of addition and multiplication is needed in order to
justify the analysis of sequences generated by FCSRs. The proof for the $N$-adic
integers being a commutative ring with an identity is given here.
  
% Definition of N-adic integers
\begin{definition}
\label{def:N-adic}
  Let $N$ be an integer. Then the infinite integer sequence $\xn$
  determines a {\em $N$-adic integer} $\alpha$, or $\xn \rightarrow \alpha$, if
\begin{equation} \label{eq:seq}
  x_{i+1} \equiv x_i \pmod{N^{i+1}} \ \ \ \forall i \geq 0.
\end{equation}
  Two sequences $\xn$ and $\{x_n'\}$ determine the same $N$-adic integer
  if 
\begin{equation} \label{eq:equiv}
  x_i \equiv x_i' \pmod{N^{i+1}}\ \ \ \forall i \geq 0.
\end{equation}
  The {\em set of all $N$-adic integers} will be denoted by $\zzzn$.
\end{definition}

\par Ordinary integers will be called {\em rational integers} and each
rational integer $x$ is associated with a $N$-adic integer, determined
by the sequence \{$x,\ x, \ \dots, \ x,\ \dots$\}.
	
% example of equivalent sequences in Zp
\begin{example} \label{ex:equiv-seq}
  Let $\xn \rightarrow \alpha \in \zzz_3$. Then the first 5 terms of
  $\xn$ may look something like:
  \begin{align*}
    \xn = \{&1 \ , \ 1+2\cdot3 \ , \ 1+2\cdot3+1\cdot3^2 \ , \\
    &1+2\cdot3+1\cdot3^2 \ , \ 1+2\cdot3+1\cdot3^2+1\cdot3^4 \ , \ \dots\} \\
        = \{&1,7,16,16,97,\dots\}
  \end{align*}
  Then equivalent sequences to $\xn$ could begin differently for the
  first few terms:
  \begin{align*}
    \yn &= \{4,25,16,178,583,\dots\} \\
    \zn &= \{-2,-47,232,-308,97,\dots\}
  \end{align*}
  The sequences for $\yn$ and $\zn$ satisfy equation (\ref{eq:seq})
  for the first 5 terms, so they could be $N$-adic integers up to this
  point. Also, both are equivalent to $\xn$ according to the
  equivalence defined in equation (\ref{eq:equiv}).
  \begin{align*}
    &1 \equiv 4 \equiv 2 \pmod 3 \\
    &7 \equiv 25 \equiv -47 \pmod{3^2} \\
    &16 \equiv 16 \equiv 232 \pmod{3^3} \\
    &16 \equiv 178 \equiv -308 \pmod{3^4} \\
    &97 \equiv 583 \equiv 97 \pmod{3^5}
  \end{align*}
  Therefore $\xn,\yn,\zn \rightarrow \alpha$.
\end{example}

\par Because there are infinitely many sequence representations for any $N$-adic
integer, it is useful to define a canonical sequence to be used when
writing $N$-adic integers as sequences.
  
\begin{definition}
\label{def:canon}
  For a given $N$-adic integer $\alpha$, a given sequence $\an$ with
  the properties:
  \begin{enumerate}[i.]
    \item $\an \rightarrow \alpha$
    \item $\an=\{a_0, \ a_0+a_1\cdot N, \ \dots, \ a_0+\dots+a_i\cdot N^i, \ \dots\} : 0 \leq a_i < N \ \ \ \forall i \geq 0$
  \end{enumerate}
  will be called {\em canonical}. The number $a_0a_1a_2 \dots a_i \dots$
  is the {\em digit representation} of $\alpha$.
\end{definition}

% Insert example of a p-adic integer written in canonical form
\begin{example} \label{ex:canon}
  In Example \ref{ex:equiv-seq}, the sequence $\xn$ was a canonical
  sequence that determined the $N$-adic integer $\alpha$. A few more examples
  of canonical sequences determining 7-adic integers are given here:\\
  \\
 $\beta = 3164\dots$, then the canonical sequence $\{b_n\} \rightarrow \beta$ is
  \begin{align*}
    \{b_n\} &= \{3, \ 3+1\cdot7, \ 3+1\cdot7+6\cdot7^2, \ 3+1\cdot7+6\cdot7^2+4\cdot7^3, \ \dots\}\\
            &= \{3,10,304,1676,\dots\}
  \end{align*}
  $\gamma = 0164\dots$, then the canonical sequence $\{c_n\} \rightarrow \gamma$ is
  \begin{align*}
    \{c_n\} &= \{0, \ 1\cdot7, \ 1\cdot7+6\cdot7^2, \ 1\cdot7+6\cdot7^2+4\cdot7^3, \ \dots\}\\
            &= \{0,7,301,1673,\dots\}
  \end{align*}
  $\delta = 5031\dots$, then the canonical sequence $\{d_n\} \rightarrow \delta$ is
  \begin{align*}
    \{d_n\} &= \{5, \ 5, 5+3\cdot7^2, \ 5+3\cdot7^2+1\cdot7^3, \ \dots\}\\
            &= \{5,5,152,495,\dots\}.
  \end{align*}
\end{example}

\begin{definition}
  Addition and multiplication in $\zzzn$ are done term by term. \\
  Let $\alpha,\beta \in \zzzn$ and $\xn \rightarrow \alpha, \yn \rightarrow \beta$. Then,
  \begin{align*}
    \xn+\yn     &:= \{x_0+y_0, x_1+y_1, \dots\} \rightarrow \alpha+\beta\\
    \xn\cdot\yn &:= \{x_0 \cdot y_0, x_1 \cdot y_1, \dots\} \rightarrow \alpha \cdot \beta\\ 
  \end{align*}

  Define $\{0,0,0,\dots\} \rightarrow 0 \in \zzzn$ and $\{1,1,1,\dots\} \rightarrow 1 \in \zzzn$
\end{definition}

\begin{lemma}\label{lem:identity}
  For $\alpha \in \zzzn$, $\alpha+0=0+\alpha=\alpha$ and $1\cdot\alpha=\alpha\cdot1=\alpha$.
\end{lemma}
\begin{proof}
  Let $\xn\rightarrow\alpha\in\zzzn$.
  \begin{align*}
    \xn+\{0,0,\dots\}&=\{x_0+0,x_1+0,\dots,x_i+0,\dots\}\\
                     &=\{x_0,\dots,x_i,\dots\}.\\
    \{0,0,\dots\}+\xn&=\{0+x_0,0+x_1,\dots,0+x_i,\dots\}\\
                     &=\{x_0,\dots,x_i,\dots\}.
  \end{align*}
  $\xn=\xn+\{0,0,\dots\}=\{0,0,\dots\}+\xn$ implies $\alpha=\alpha+0=0+\alpha$.
  Therefore, the additive identity in $\zzzn$ is 0.

  \begin{align*}
    \xn\cdot\{1,1,\dots\}&=\{x_0\cdot 1,x_1\cdot 1,\dots,x_i\cdot 1,\dots\}\\
                         &=\{x_0,\dots,x_i,\dots\}.\\
    \{1,1,\dots\}\cdot\xn&=\{1\cdot x_0,1\cdot x_1,\dots,1\cdot x_i,\dots\}\\
                         &=\{x_0,\dots,x_i,\dots\}.
  \end{align*}
   $\xn=\xn\cdot\{1,1,\dots\}=\{1,1,\dots\}\cdot\xn$ implies $\alpha=\alpha\cdot1=1\cdot\alpha$.
  Therefore, the multiplicative identity in $\zzzn$ is 1.
\end{proof}   
% Insert examples of multiplication and addition

\par Finally, this paper defines a {\em ring} accroding to Fine, Gaglione, and Roesenberger and
shows that $\zzzn$ is a {\em commutative ring with an identity}.
\begin{definition}
\label{def:ring}
  A {\em ring} is a set $R$ with two binary operations defined on it. These are usually
  called addition denoted by +, and multiplication denoted by $\cdot$ or juxtaposition,
  satisfying the following six axioms:
  \begin{enumerate}
    \item Addition is commutative: $a+b=b+a \ \ \forall a,b \in R$.
    \item Addition is associative: $a+(b+c)=(a+b)+c \ \ \forall a,b,c \in R$.
    \item There exists an additive identity, denoted by 0, such that $a+0=a \ \ \forall a \in R$.
    \item $\forall a \in R$ there exists an additive inverse, denoted by $-a$, such
      that $a+(-a)=0$.
    \item Multiplication is associative: $a(bc)=(ab)c \ \forall a,b,c \in R$
    \item Multiplication is left and right distributive over addition:
      \begin{align*}
        a(b+c)&=ab+ac \\
        (b+c)a&=ba+ca
      \end{align*}
  \end{enumerate}
  If it is also true that
  \begin{enumerate}
      \setcounter{enumi}{6}
    \item Multiplication is commutative: $ab=ba \ \ \forall a,b \in R$, then
      $R$ is a {\em commutative ring}.
  \end{enumerate}
  Further if
  \begin{enumerate}
      \setcounter{enumi}{7}
    \item There exists a multiplicative identity denoted by 1 such that
      $a \cdot 1=a$ and $1 \cdot a=a \ \ \forall a \in R$, then $R$ is a
      {\em ring with an identity}.
  \end{enumerate}
  If $R$ satisfies all eight properties, then $R$ is a {\em commutative ring with
  an identity.}
\end{definition}

\begin{theorem}
  $\zzzn$ is a commutative ring with an identity.
\end{theorem}
% p-adic integer ring proof
\begin{proof}
  Let $\xn,\yn,\zn$ determine $\alpha,\beta,\gamma\in\zzzn$ respectively. Then
  \begin{enumerate}
    \item {\em Commutativity of Addition}
      \begin{align*}
            \xn+\yn&=\{x_0+y_0,\dots,x_i+y_i,\dots\} \\
                   &=\{y_0+x_0,\dots,y_i+x_i,\dots\} \\
                   &=\yn+\xn.
      \end{align*}
      $\xn+\yn \rightarrow \alpha+\beta$ and $\xn+\yn=\yn+\xn \rightarrow \beta+\alpha$.
      Therefore, by Definition \ref{def:N-adic}, $\alpha+\beta=\beta+\alpha$.
    \item {\em Associativity of Addition}
      \begin{align*}
        \xn+(\yn+\zn)&=\xn+\{y_0+z_0,\dots,y_i+z_i,\dots\} \\
                     &=\{x_0+(y_0+z_0),\dots,x_i+(y_i+z_i),\dots\} \\
                     &=\{(x_0+y_0)+z_0,\dots,(x_i+y_i)+z_i,\dots\} \\
                     &=\{x_0+y_0,\dots,x_i+y_i,\dots\}+\zn \\
                     &=(\xn+\yn)+\zn.
      \end{align*}
      Therefore, $\alpha+(\beta+\gamma)=(\alpha+\beta)+\gamma$.
    \item {\em Existence of the Additive Identity}
      \\ \\
      By Lemma \ref{lem:identity}, $0$ is the additive identity.
    \item {\em Existence of Additive Inverses}
      \\ \\
      Define $-\xn = \{p-x_0,p^2-x_1,\dots,p^{i+1}-x_i,\dots\} \rightarrow -\alpha$.
      Then
      \begin{align*}
        \xn+(-\xn)&=\{x_0+p-x_0,x_1+p^2-x_1,\dots,x_i+p^{i+1}-x_i,\dots\} \\
                  &=\{p,p^2,\dots,p^{i+1},\dots\} \\
                  &\equiv\{0,0,\dots\} \\
                  &=0.
      \end{align*}
      Therefore, $\alpha+(-\alpha)=0$.
    \item {\em Associativity of Multiplication}
      \begin{align*}
        \xn(\yn\zn)&=\xn\{y_0z_0,\dots,y_iz_i,\dots\} \\
                   &=\{x_0(y_0z_0),\dots,x_i(y_iz_i),\dots\} \\
                   &=\{(x_0y_0)z_0,\dots,(x_iy_i)z_i,\dots\} \\
                   &=\{x_0y_0,\dots,x_iy_i,\dots\}\zn \\
                   &=(\xn\yn)\zn.
      \end{align*}
      Therefore, $\alpha(\beta\gamma)=(\alpha\beta)\gamma$.
    \setcounter{enumi}{6}
    \item {\em Commutativity of Multiplication}
      \begin{align*}
        \xn\yn&=\{x_0y_0,\dots,x_iy_i,\dots\} \\
              &=\{y_0x_0,\dots,y_ix_i,\dots\} \\
              &=\yn\xn.
      \end{align*}
      Therefore, $\alpha\beta=\beta\alpha$.
    \setcounter{enumi}{5}
    \item {\em Left and right distributivity of multiplication over addition}
      \begin{align*}
        \xn(\yn+\zn)&=\xn\{y_0+z_0,\dots,y_i+z_i,\dots\} \\
                    &=\{x_0(y_0+z_0),\dots,x_i(y_i+z_i),\dots\} \\
                    &=\{x_0y_0+x_0z_0,\dots,x_iy_i+x_iz_i,\dots\} \\
                    &=\xn\yn+\xn\zn. \\
      \end{align*}
      By commutativity of multiplication,
      \begin{align*}
        (\yn+\zn)\xn&=\xn(\yn+\zn)\\
                    &=\xn\yn+\xn\zn\\
                    &=\yn\xn+\zn\xn.
      \end{align*}
      Therefore, $\alpha(\beta+\gamma)=\alpha\beta+\alpha\gamma$ and
      $(\beta+\gamma)\alpha=\beta\alpha+\gamma\alpha$.
    \setcounter{enumi}{7}
    \item {\em Existence of a multiplicative identity}
      \\ \\
      By Lemma \ref{lem:identity}, 1 is the multiplicative identity.
      \\ \\
      Properties 1 through 8 from Definition~\ref{def:ring} are satisfied,
      so $\zzzn$ is a commutative ring with an identity.
  \end{enumerate}
\end{proof}

\subsection{Units}
Here is a short of proof of what $N$-adic integer units look like.

\begin{theorem}\label{thm:units}
  An $N$-adic integer $\alpha$, which is determined by a sequence $\xn$, is
  a unit if and only if $x_0$ is relatively prime to $N$.
\end{theorem}
\begin{proof}
  Let $\alpha$ be a unit. Then there is an $N$-adic integer $\beta$ such that
  $\alpha\beta=1$. If $\beta$ is determined by the sequence $\yn$, then
  \begin{equation}\label{eq:units}
    x_iy_i\equiv1\pmod{N^{i+1}} \ \ \forall i \geq 0.
  \end{equation}
  In particular, $x_0y_0\equiv1\pmod N$ and hence $x_0\not\equiv0\bmod N$.
  Thus, $x_0$ is relatively prime to $N$.
  Conversely, let $x_0$ be relatively prime to $N$. Then $x_0\not\equiv0\pmod{N}$.
  From (\ref{eq:seq})
  \begin{align*}
    x_1 &\equiv x_0 \pmod N\\
    &\vdots \\
    x_{i+1} &\equiv x_i \pmod{N^i}. 
  \end{align*}
  Working backward, $x_{i+1} \equiv x_i \equiv \dots \equiv x_1 \equiv x_0 \pmod N$.
  Thus, $x_i$ is relatively prime to $p \ \ \forall i\geq0$, which implies
  $x_i$ is relatively prime to $N^{i+1}$. Consequently, $\forall i\geq0 \
  \exists y_i$ such that $x_iy_i \equiv 1 \pmod{N^{i+1}}$. Since
  $x_{i+1} \equiv x_i \pmod p^i$ and $x_{i+1}y_{i+1} \equiv x_iy_i \pmod{N^i}$.
  Then, $y_{i+1} \equiv y_i \pmod{N^i}$. Therefore the sequence $\yn$ determines
  some $N$-adic integer $\beta$. Because $x_iy_i \equiv 1 \pmod{N^{i+1}} \ \ \forall i \geq 0$,
  $\alpha\beta=1$. This means $\alpha$ is a unit.
\end{proof}

\par From this theorem it follows that a rational integer $a\in\zzzn$ is a unit if
and only if $a$ is relatively prime to $N$. If this condition holds, then $a^{-1}\in\zzzn$.
Then for any rational integer $b\in\zzzn$, $b/a=a^{-1}b\in\zzzn$. Here is a corollary
about the digit representation of $N$-adic integers that immediately follows.

\begin{corollary}
	The $N$-adic integer $\alpha$ is a unit if and only if the first digit of $\alpha$ is non-zero.
\end{corollary}

\subsection{Constructing Sequences Determing $\frac{b}{a}$ in $\zzzn$}

\par For any rational number $b/a$, $a$ relatively prime to $N$, there
exists a sequence $\xn \rightarrow b/a \in \zzzn$. At this point, it is worth
using the digit representation for integers in $\zzzn$. So
$\xn=\{x_0, \ x_0+x_1N, \ \dots, \ x_0+\dots+x_iN^i, \ \dots\}$
and $b/a = x_0x_1\dots x_i\dots$. Rather than finding $a^{-1}\pmod N^{i+1}$ to
determine each $x_i$, it is not too difficult for every $i$ to find $\sum_{k=0}^ix_kN^k$
such that
\begin{equation}\label{eq:seq-rational}
  b \equiv a\sum_{k=0}^ix_kN^k \pmod{N^{i+1}}.
\end{equation}
Then, 
\begin{equation}
  x_i = \frac{\sum_{k=0}^ix_kN^k - \sum_{k=0}^{i-1}x_kN^k}{N^k}.
\end{equation}

\par Nearly all of the digits for any rational number in $\zzzn$ can also
be found using powers of $N^{-1}$, which is much simpler to analyze
than the brute force search for the digits mentioned above.

\begin{theorem}\label{thm:10}
  Let $u_0,q,N\in\zzz$, where $q$ is relatively prime to $N$,
  $\lvert u_0 \rvert < q$, and $q=-q_0+\sum_{i=1}^{r}q_iN^i$ for
  $0 \leq q_i < N$. Define $\alpha = u_0/q \in \zzzn$
  such that $\alpha = \sum_{i=0}^{\infty}a_iN^i$ for $0 \leq a_i < N$.
  Also, define $u_k\in\zzz$ such that $u_k/q = \sum_{i=k}^{\infty}a_iN^{i-k} \in \zzzn$
  and $\gamma \equiv N^{-1} \pmod q$. Then,
  there exist $u_k$ for every $k\geq0$ such that
  \begin{equation}\label{eq:ak}
    a_k \equiv q^{-1}u_k \pmod N.
  \end{equation}
  If $-q<u_0<0$, then $u_k \in \{-q,\dots,-1\}$ for $k\geq0$.
  Otherwise, for $k>\lfloor \log_N(q) \rfloor=r$, $u_k \in \{-q,\dots,-1\}$
  \\ \\
  Let $\omega \in \{-q,\dots,-1\}$ such that $\omega \equiv \gamma^k u_0 \pmod q$.
  Then for $k>\lfloor \log_N(q) \rfloor=r$, or if $-q<u_0<0$, then $k\geq0$,
  \begin{equation}\label{eq:ak-omega}
    a_k \equiv q^{-1}\omega \pmod N.
  \end{equation}
\end{theorem}
\noindent \\ 
\begin{proof}
%  Let $u_i/q = \sum_{k=i}^{\infty}a_kN^{k-i} \in \zn$. Then,
%  \begin{align*}
%    \frac{u_0}{q} &= a_0+a_1N+a_2N^2+\dots \\ 
%                  &\vdots \\ 
%    \frac{u_i}{q} &= a_i+a_{i+1}N+a_{i+2}N^2+\dots \\
%                  &\vdots
%  \end{align*}
%  It follows that
  Write $u_0/q$ in terms of $u_k$.
  \begin{align}
    \frac{u_0}{q} &= a_0 + N\frac{u_1}{q} = a_0 + a_1N + N^2\frac{u_2}{q} = \dots \nonumber \\
                  &= \sum_{i=0}^{k-1}a_iN^i + N^k\frac{u_k}{q} \ \ \forall k \geq 1. \label{eq:u0/q}
  \end{align}
%  Rewrite (\ref{eq:u0/q}) to be
%  \begin{equation}\label{eq:u0/q-rw}
%    u_0 = q(\sum_{i=0}^{k-1}a_iN^i) + N^ku_k \ \ \forall k \geq 1.
%  \end{equation}
%  For $k=1$, $u_0=qa_0+Nu_1$. Therefore $a_0 \equiv q^{-1}u_0 \pmod N$. This
%  completes half of the theorem.\\
  Rewrite (\ref{eq:u0/q}) to be
  \begin{equation}\label{eq:u0/q-rw}
    p^ku_k=u_0-q\bigg(\sum_{i=0}^{k-1}a_ip^i\bigg) \ \ \forall k \geq 1
  \end{equation}
  Then $\lvert u_0 \rvert < q$ and $0 \leq a_i < p$ from the assumptions
  and equation (\ref{eq:u0/q-rw}). These imply for all $k \geq 1$,
  $\lvert u_0 \rvert = \lvert q\sum_{i=0}^{k-1}a_ip^i + p^ku_k \rvert < q$.
  Then,
  \begin{align*}
    -q-q\sum_{i=0}^{k-1}a_ip^i < p^k&u_k < q-q\sum_{i=0}^{k-1}a_ip^i \\
    \Rightarrow -q\bigg(\frac{1+\sum_{i=0}^{k-1}a_ip^i}{p^k}\bigg) < \ &u_k < q\bigg(\frac{1-\sum_{i=0}^{k-1}a_ip^i}{p^k}\bigg).
  \end{align*}
  $u_k$ may only greater than zero when $\frac{1-\sum_{i=0}^{k-1}a_ip^i}{p^k}$ greater than zero.
  This only occurs when the sequence $[a_0,\dots,a_j] = [0,\dots,0]$ for $j\geq0$.
  Such a sequence occurs if and only if $u_0\geq0$ and $u_0 \equiv 0 \pmod p^i$ for $0\leq i \leq j$, $j\geq0$.
  This is clear from the construction of $p$-adic sequences for
  rational numbers. Therefore $u_k$ may only be greater than zero
  if $u_0\geq0$ and $u_0 \equiv 0 \pmod p^i$ for $0\leq i \leq j$, $j\geq0$.
  The lower bound is greater than $-q$. This clear because
  $\frac{1+\sum_{i=0}^{k-1}a_ip^i}{p^k}\leq1$.
  Therefore,
  \begin{equation*}
    -q < u_k < 0 \ for \ -q<u_0<0.
  \end{equation*}
  If $0\geq u_0<q$, then the upper bound remains unchanged.
  \begin{equation*}
    -q < u_k < q\bigg(\frac{1-\sum_{i=0}^{k-1}a_ip^i}{p^k}\bigg) \ {\rm for} \ 0\leq u_0 < q
  \end{equation*}
  There is still work to be done on the upper bound.
  \begin{align*}
                &0 \leq \sum_{i=0}^{k-1}a_ip^i < p^k \ for \ k\geq1 \\
    \Rightarrow &-q(\sum_{i=0}^{k-1}a_ip^i) \leq 0 \\
    \Rightarrow \ &u_0-q(\sum_{i=0}^{k-1}a_ip^i)<q \\
    \Rightarrow \ &p^ku_k<q \\
    \Rightarrow \ &u_k<\frac{q}{p^k}.
  \end{align*}
  For $k>\lfloor\log_p(q)\rfloor=r$, $\lvert q/p^k \rvert < 1$. Therefore,
  $-q < u_k < 0 \ {\rm for} \ 0\leq u_0 < q \ {\rm and} \ k>r$.
  Further lowering the upperbound, if $u_k=0$, then $u_0/q=\sum_{i=0}^{k-1}a_ip^i+0$.
  This implies $u_0/q$ is a rational integer, which is not true. Noting finally
  that $u_k$ must be an integer. If $\lvert u_0 \rvert<q$ and $u_0<0$, or
  $\lvert u_0 \rvert<q$, $u_0\geq0$, and $k>\lfloor \log_p(q) \rfloor=r$, then
  \begin{equation*}
    u_k \in \{-q,\dots,-1\}.
  \end{equation*}
  It has now been shown for certain restrictions $u_k$ belongs to a specific
  set of representatives for the residue classes of $\zzz/(q)$. Define 
  $\gamma \equiv p^{-1} \pmod q$. Reducing (\ref{eq:u0/q-rw}) modulo $q$ shows that
  \begin{equation}
    u_k \equiv \gamma u_{k-1} \pmod q.
  \end{equation}
  Since this is true for all k greater than or equal to 1, it is clear that
  \begin{equation}\label{eq:uk-mod-q}
    u_k \equiv \gamma^ku_0 \pmod q.
  \end{equation}
  Reducing (\ref{eq:u0/q-rw}) modulo $p$ shows that
  \begin{equation}\label{eq:ak-mod-p}
    a_k \equiv q^{-1}u_k \pmod p.
  \end{equation}
  Define $\omega \equiv \gamma^k u_0 \pmod q$, and $\rho \equiv q^{-1} \bmod p$.
  Finally, if $\lvert u_0 \rvert<q$ and $u_0<0$, or
  $\lvert u_0 \rvert<q$, $u_0\geq0$, and $k>\lfloor \log_p(q) \rfloor=r$, then
  \begin{equation}\label{eq:ak-done}
    a_k \equiv \rho\omega \pmod p.
  \end{equation}
\end{proof}
\begin{corollary}\label{cor:aj}
  Let $0\leq u_0 < q$. Define $j$ to be the greatest integer such that
  $u_0 \equiv 0 \pmod p^j$. Then the following are true:
  \begin{enumerate}[i.]
    \item $j\leq\lfloor\log_p{q}\rfloor = r$
    \item $[a_0,\dots,a_{j-1}]=[0,\dots,0]$
    \item $u_k>0$ for $k=j$
    \item $u_k \not\equiv 0 \pmod p$
  \end{enumerate}
\end{corollary}
\par The results of this corollary are straightforward.

\par Theorem \ref{thm:10} shows that for $-q<u_0<0$, there is a sequence of
numerators $\{u_k\}$ directly related to the sequence of digits $\{a_k\}$
for $u_0/q\in\zn$. This is provides a more powerful tool for the analysis
of the sequences generated by AFSRs. The results shown here fill in the gaps of
the incorrect proof shown in Theorem 10 of Klapper and Xu's paper.

