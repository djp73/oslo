\begin{abstract}
	Stream ciphers make use of pseudorandom sequences to create
	near perfect secrecy. Though, if the attacker finds a way to
	hop onto the keystream, then the whole cipher is broken. This
	is the exact problem that exists with the feedback with carry
	shift register (FCSR). This paper will investigate how bent functions
	can be used to produce bent sequences that possess non-linearity
	properties that are desirable to resist regsiter synthesis
	attacks. Also, the paper considers the 2-adic span of bent
	sequences which say something about the size of the FCSR necessary
	to produce a given bent sequence.
\end{abstract}
