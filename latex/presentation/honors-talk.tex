\documentclass{beamer}
\usetheme{default}
\usepackage{amsmath}
\usepackage{amssymb}
\usepackage{amsthm}
\usepackage{graphicx}
\usepackage{verbatim}

% Definitions
\def\qqq{\mathbb{Q}}
\def\rrr{\mathbb{R}}
\def\zzz{\mathbb{Z}}
\def\fff{\mathbb{F}}
\def\gftwo{\mathbb{F}_2}
\def\zzzp{\mathbb{Z}_p}
\def\zzzn{\mathbb{Z}_N}
\def\zg{\mathbb{Z}_g}
\def\nnn{\mathbb{N}}
\def\BF{\mathcal{BF}}
\def\xn{(x_n)}
\def\yn{(y_n)}
\def\zn{(z_n)}
\def\an{(a_n)}
\def\Chi{\raisebox{2pt}{$\chi$}}
% \def\qed{$\Box}
% \newcounter{padic}

\begin{document}
\title{Bent Sequences and Feedback with Carry Shift Registers}
\author{Charles Celerier}
\frame{\titlepage}
%\frame{\tableofcontents}

\section{Introduction}
\subsection{Cryptography}
\begin{frame}{Cryptography}
  %% The Basic Communication Scenario for Cryptography
  \setlength{\unitlength}{1mm}
  \begin{picture}(0,0)(-55,0)
    \put(-50,-1.3){\framebox{Alice}}
    \put(0,0){\circle*{2}}
    \put(-39.9,0){\vector(4,0){15}}
    \put(-24.9,-1){\framebox{Encrypt}}
    \put(-10,0){\line(4,0){9}}
    \put(-38.4,2){\makebox{\footnotesize plaintext}}
    \put(-17.45,15){\vector(0,-4){12.2}}
    \put(-24,20){\makebox{\footnotesize encryption}}
    \put(-19.5,16.5){\makebox{\footnotesize key}}
    \put(-6.5,2){\makebox{\footnotesize ciphertext}}
    \put(1,0){\vector(4,0){9}}
    \put(0,-1){\vector(0,-4){12.2}}
    \put(-4.2,-17){\framebox{Eve}}
    \put(0,-13.2){\vector(0,4){12.2}}
    \put(10,-1){\framebox{Decrypt}}
    \put(24.9,0){\vector(4,0){15}}
    \put(26.1,2){\footnotesize plaintext}
    \put(39.9,-1.3){\framebox{Bob}}
    \put(17.45,15){\vector(0,-4){12.2}}
    \put(11,20){\makebox{\footnotesize decryption}}
    \put(15.5,16.5){\makebox{\footnotesize key}}
  \end{picture}
\end{frame}
\begin{frame}{Kerckhoff's Principle}
  ``In assessing the security of a cryptosystem, one should always
  assume the enemy knows the method being used.''
\end{frame}

\subsection{Stream Ciphers}
\begin{frame}{$\gftwo$ or ``GF two''}
  \begin{center}
  Binary Operations for $\gftwo$
\end{center}
  \begin{table}[h!]\label{tab:GF(2)}
    \centering
    \begin{tabular}{|c|}
      \hline
      XOR\\
      \hline
      $0\oplus0:=0$\\
      $0\oplus1:=1$\\
      $1\oplus0:=1$\\
      $1\oplus1:=0$\\
      \hline
    \end{tabular}
  \end{table}
\end{frame}

\begin{frame}{Stream Ciphers}
  \begin{center}
    Encryption
  \end{center}
  \begin{tabular}{c c l}
    \pause & 0101001101000001010100110100110101000011 & \text{plaintext}\\
    $\oplus$ & 0001100000001011000111110001110100010010 & \text{key}\\
    \hline 
    \pause
    & 0100101101001010010011000101000001010001 & \text{ciphertext}
  \end{tabular}
\end{frame}
\begin{frame}{Stream Ciphers}
  \begin{center}
  Decryption
\end{center}
  \begin{tabular}{c c l}
    & 0100101101001010010011000101000001010001 & \text{ciphertext}\\
    $\oplus$ & 0001100000001011000111110001110100010010 & \text{key}\\
    \hline 
    & 0101001101000001010100110100110101000011 & \text{plaintext}
  \end{tabular}
\end{frame}

\begin{frame}{Why use stream ciphers?}
  \begin{itemize}
    \item plaintext length is not always known
    \item fast and easy to implement in hardware
    \item near one-time-pad security
  \end{itemize}
\end{frame}

\subsection{What is a pseudorandom sequence?}
\begin{frame}{What is a pseudorandom sequence?}
  \begin{itemize}
    \item[R1.] uniform distribution \[|\sum_{n=1}^p(-1)^{a_n}|\leq1\]
    \item[R2.] \[\frac{1}{2^i}\text{ of the runs have length } i\]
    \item[R3.] low auto-correlation, \[C(\tau)
      =\frac{\sum_{n=1}^p(-1)^{a_n+a_{n+\tau}}}{p}\]
  \end{itemize}
\end{frame}

\begin{frame}{Example}
  \begin{align*}
    a = [1,0,0,1,1,1,0]\\
  \end{align*}
  \setlength{\unitlength}{1mm}
  \begin{picture}(90,40)(-40,-25)
    %% State of the register
    \put(0,0){\framebox(10,10){0}}
    \put(10,0){\framebox(10,10){1}}
    \put(20,0){\framebox(10,10){1}}
    \put(30,5){\vector(4,0){10}}
    %% Taps
    \put(0,-13){\makebox(10,10){1}}
    \put(5,-8.5){\circle{9}}
    \put(10,-13){\makebox(10,10){1}}
    \put(15,-8.5){\circle{9}}
    \put(20,-13){\makebox(10,10){0}}
    \put(25,-8.5){\circle{9}}
    %% Lines connecting Taps and the State of the Register
    \multiput(5,-0.2)(10,0){3}{\line(0,-6){4}}
    %% Lines from Taps to Summer
    \put(5,-12.8){\line(0,-6){7.2}}
    \put(15,-12.8){\line(0,-6){7.2}}
    \put(25,-12.8){\line(0,-6){7.2}}
    \put(26.5,-20){\line(-4,0){39.5}}
    \put(5,-20){\circle{3}}
    \put(5,-20){\line(0,-4){1.5}}
    \put(15,-20){\circle{3}}
    \put(15,-20){\line(0,-4){1.5}}
    \put(25,-20){\circle{3}}
    \put(25,-20){\line(0,-4){1.5}}
    \put(-13,-20){\line(0,0){25}}
    \put(-13,5){\vector(4,0){13}}
  \end{picture}
\end{frame}

\begin{frame}{LFSR}
  \begin{center}
    Linear Feedback Shift Register
  \end{center}
  \setlength{\unitlength}{1mm}
  \begin{picture}(90,40)(-45,-25)
    %% State of the register
    \put(0,0){\framebox(10,10){$x_{n-1}$}}
    \put(10,0){\framebox(10,10){$x_{n-2}$}}
    \put(20,0){\framebox(10,10){$\dots$}}
    \put(30,0){\framebox(10,10){$x_{1}$}}
    \put(40,0){\framebox(10,10){$x_{0}$}}
    \put(50,5){\vector(4,0){10}}
    %% Taps
    \put(0,-13){\makebox(10,10){$q_1$}}
    \put(5,-8.5){\circle{9}}
    \put(10,-13){\makebox(10,10){$q_2$}}
    \put(15,-8.5){\circle{9}}
    \put(20,-13){\makebox(10,10){$\dots$}}
    \put(30,-13){\makebox(10,10){$q_{n-1}$}}
    \put(35,-8.5){\circle{9}}
    \put(40,-13){\makebox(10,10){$q_n$}}
    \put(45,-8.5){\circle{9}}
    %% Lines connecting Taps and the State of the Register
    \multiput(5,-0.2)(10,0){2}{\line(0,-6){4}}
    \multiput(35,-0.2)(10,0){2}{\line(0,-6){4}}
    %% Summer
    \put(-20,-25){\framebox(10,10){\Large $\sum$}}
    %% Lines from Taps to Summer
    \put(5,-12.8){\line(0,-6){3.5}}
    \put(15,-12.8){\line(0,-6){5.5}}
    \put(35,-12.8){\line(0,-6){8.5}}
    \put(45,-12.8){\line(0,-6){10.5}}
    \put(5,-16.3){\vector(-4,0){15}}
    \put(15,-18.3){\vector(-4,0){25}}
    \put(35,-21.3){\vector(-4,0){45}}
    \put(45,-23.3){\vector(-4,0){55}}
    %% Lines from Summer
    \put(-13,-15){\line(0,0){20}}
    \put(-13,5){\vector(4,0){13}}
    %% mod N
    \put(-13,5){\makebox(13,5){mod 2}}
  \end{picture}
\end{frame}

\begin{frame}{FCSR}
  \begin{center}
    Feedback with Carry Shift Register
  \end{center}
  \setlength{\unitlength}{1mm}
  \begin{picture}(90,40)(-45,-25)
    %% State of the register
    \put(0,0){\framebox(10,10){$x_{n-1}$}}
    \put(10,0){\framebox(10,10){$x_{n-2}$}}
    \put(20,0){\framebox(10,10){$\dots$}}
    \put(30,0){\framebox(10,10){$x_{1}$}}
    \put(40,0){\framebox(10,10){$x_{0}$}}
    \put(50,5){\vector(4,0){10}}
    %% Taps
    \put(0,-13){\makebox(10,10){$q_1$}}
    \put(5,-8.5){\circle{9}}
    \put(10,-13){\makebox(10,10){$q_2$}}
    \put(15,-8.5){\circle{9}}
    \put(20,-13){\makebox(10,10){$\dots$}}
    \put(30,-13){\makebox(10,10){$q_{n-1}$}}
    \put(35,-8.5){\circle{9}}
    \put(40,-13){\makebox(10,10){$q_n$}}
    \put(45,-8.5){\circle{9}}
    %% Lines connecting Taps and the State of the Register
    \multiput(5,-0.2)(10,0){2}{\line(0,-6){4}}
    \multiput(35,-0.2)(10,0){2}{\line(0,-6){4}}
    %% Summer
    \put(-20,-25){\framebox(10,10){\Large $\sum$}}
    %% Lines from Taps to Summer
    \put(5,-12.8){\line(0,-6){3.5}}
    \put(15,-12.8){\line(0,-6){5.5}}
    \put(35,-12.8){\line(0,-6){8.5}}
    \put(45,-12.8){\line(0,-6){10.5}}
    \put(5,-16.3){\vector(-4,0){15}}
    \put(15,-18.3){\vector(-4,0){25}}
    \put(35,-21.3){\vector(-4,0){45}}
    \put(45,-23.3){\vector(-4,0){55}}
    %% Lines from Summer
    \put(-13,-15){\line(0,0){20}}
    \put(-13,5){\vector(4,0){13}}
    \put(-17,-15){\line(0,0){20}}
    \put(-17,5){\vector(-4,0){13}}
    %% Memory cell z
    \put(-40,0){\framebox(10,10){$z$}}
    %% Lines from memory cell to summer
    \put(-35,0){\line(0,-6){20}}
    \put(-35,-20){\vector(4,0){15}}
    %% div 2 mod 2
    \put(-13,5){\makebox(13,5){mod $2$}}
    \put(-29.5,5){\makebox(13,5){div $2$}}
  \end{picture}
\end{frame}

\begin{frame}
  \begin{itemize}
    \item 2-adic integers
    \item Boolean functions
    \item Bent Sequences
  \end{itemize}
\end{frame}

\section{2-adic integers}
\begin{frame}{2-adic integers}
  10101010101010101010101010\ldots\\
  \pause
  \begin{definition}
    The infinite integer sequence $\xn$
    determines a {\bf 2-adic integer} $\alpha$, or
    $\xn \rightarrow \alpha$, if
    \begin{equation} \label{eq:seq}
    x_{i+1} \equiv x_i\pmod{2^{i+1}} \ \ \ \forall i \geq 0.
    \end{equation}
    Two sequences $\xn$ and $(x_n')$ determine the same $2$-adic integer if 
  \begin{equation} \label{eq:equiv}
    x_i \equiv x_i' \pmod{2^{i+1}}\ \ \ \forall i \geq 0.
  \end{equation}
    The {\bf set of all 2-adic integers} will be denoted by $\zzz_2$.
  \end{definition}
\end{frame}

\begin{frame}{2-adic integers}
  Let $\xn \rightarrow \alpha \in \zzz_2$. Then the first 5 terms of
  $\xn$ may look something like:
  \begin{align*}
    \xn = (&\ 1\ , \\
           &\ 1+0\cdot2 \ , \\
           &\ 1+0\cdot2+1\cdot2^2 \ ,\\
           &\ 1+0\cdot2+1\cdot2^2+0\cdot2^3 \ ,\\
           &\ 1+0\cdot2+1\cdot2^2+0\cdot2^3+1\cdot2^4 \ , \ \dots\ )\\
        = (&\ 1,1,5,5,21,\dots \ )
  \end{align*}
\end{frame}

\begin{frame}{2-adic integers}
  \begin{align*}
    \alpha&=11001\cdots\\
    1&=1000\cdots\\
    2&=0100\cdots\\
    3&=1100\cdots\\
    -1&=1111\cdots\\
    1/3&=1101010101\cdots\\
    -1/3&=1010101010\cdots
  \end{align*}
\end{frame}

\begin{frame}{2-adic integers}
  \begin{definition}
    Let $\alpha=\an\in\zzz_2\setminus(0)$. If $m$ is the smallest number in
    $\nnn\cup\{0\}$ such that $a_m \not\equiv 0 \pmod 2^{m+1}$, then the {\bf
    2-adic\ valuation} of $\alpha$ is $m$, or $\log_2(\alpha)=m$. If $\alpha=0$,
    then $\log_2(\alpha)=\infty$.
  \end{definition}
  \begin{example}
    Let $\alpha=0001011101111\cdots$. Then $\log_2(\alpha)=3$.
  \end{example}
\end{frame}

\begin{frame}{FCSR}
    \[\frac{-4}{5}=00110011001100110011\cdots\]
  \setlength{\unitlength}{1mm}
  \begin{picture}(90,40)(-45,-25)
    %% State of the register
    \put(0,0){\framebox(10,10){1}}
    \put(10,0){\framebox(10,10){1}}
    \put(20,0){\framebox(10,10){$0$}}
    \put(30,0){\framebox(10,10){$0$}}
    \put(40,5){\vector(4,0){10}}
    %% Taps
    \put(0,-13){\makebox(10,10){1}}
    \put(5,-8.5){\circle{9}}
    \put(10,-13){\makebox(10,10){1}}
    \put(15,-8.5){\circle{9}}
    \put(20,-13){\makebox(10,10){0}}
    \put(25,-8.5){\circle{9}}
    \put(30,-13){\makebox(10,10){0}}
    \put(35,-8.5){\circle{9}}
    %% Lines connecting Taps and the State of the Register
    \multiput(5,-0.2)(10,0){4}{\line(0,-6){4}}
    %% Summer
    \put(-20,-25){\framebox(10,10){\Large $\sum$}}
    %% Lines from Taps to Summer
    \put(5,-12.8){\line(0,-6){3.5}}
    \put(15,-12.8){\line(0,-6){5.5}}
    \put(25,-12.8){\line(0,-6){8.5}}
    \put(35,-12.8){\line(0,-6){10.5}}
    \put(5,-16.3){\vector(-4,0){15}}
    \put(15,-18.3){\vector(-4,0){25}}
    \put(25,-21.3){\vector(-4,0){35}}
    \put(35,-23.3){\vector(-4,0){45}}
    %% Lines from Summer
    \put(-13,-15){\line(0,0){20}}
    \put(-13,5){\vector(4,0){13}}
    \put(-17,-15){\line(0,0){20}}
    \put(-17,5){\vector(-4,0){13}}
    %% Memory cell z
    \put(-40,0){\framebox(10,10){0}}
    %% Lines from memory cell to summer
    \put(-35,0){\line(0,-6){20}}
    \put(-35,-20){\vector(4,0){15}}
    %% div 2 mod 2
    \put(-13,5){\makebox(13,5){mod $2$}}
    \put(-29.5,5){\makebox(13,5){div $2$}}
  \end{picture}
\end{frame}

\section{Boolean Functions}
\subsection{GF(2)}
\begin{frame}{$\gftwo$ or ``GF two''}
  \begin{table}[h!]\label{tab:GF(2)}
    \centering
    \begin{tabular}{|c|c|}
      \hline
      XOR&AND\\
      \hline
      $0\oplus0:=0$&$0\cdot0:=0$\\
      $0\oplus1:=1$&$0\cdot1:=0$\\
      $1\oplus0:=1$&$1\cdot0:=0$\\
      $1\oplus1:=0$&$1\cdot1:=1$\\
      \hline
    \end{tabular}
  \caption{Binary Operations for $\gftwo$}
  \end{table}
\end{frame}

\begin{frame}{$\gftwo^n$ or ``GF two to the n''}
\begin{example}
  Let $a,b\in\gftwo^3$ such that $a=(1,0,1)$ and $b=(0,1,1)$ then
  \begin{align*}
    a+b      &=(1\oplus0,0\oplus1,1\oplus1)=(1,1,0) \\
  a\cdot b &=1\cdot0\oplus0\cdot1\oplus1\cdot1=1
  \end{align*}
\end{example}

\begin{fact}
  $\gftwo^n$ is a vector space.
\end{fact}
\end{frame}

\subsection{Boolean Functions}
\begin{frame}{Boolean functions in $\BF_n$}
  \begin{definition}
  \label{def:boolean-function}
    Any function $f$ defined such that 
    \begin{equation*}
      f:\gftwo^n\rightarrow\gftwo
    \end{equation*}
    is a {\bf Boolean function}. The set of all Boolean functions on $n$
    variables will be denoted by $\BF_n$.
  \end{definition}
\end{frame}

\begin{frame}{An example}
  \begin{example}
    Let $f=x_0+x_1$.
    \begin{table}
    \label{tab:truth-table}
    	\centering
      \begin{tabular}{|c|c||c|}
        \hline
        $x_0$&$x_1$&$f(x_0,x_1)$\\
        \hline
        0&0&0\\
        1&0&1\\
        0&1&1\\
        1&1&0\\
      	\hline
    	\end{tabular}
    	\caption{Truth Table of $f$}
    \end{table}
  \end{example}
\end{frame}

\subsection{Walsh Transform}
\begin{frame}{Characters of $\gftwo^n$}
\begin{definition}
  A {\bf character} $\Chi$ of a finite abelian group $G$ is a group
  homomorphism from $G$ into the multiplicative group of complex numbers.
\end{definition}
  \begin{fact}
    $\Chi_\lambda(x):=(-1)^{\lambda\cdot x}$, where $\lambda,x\in\gftwo^n$,
    is a {\bf character} of $\gftwo^n$.
  \end{fact}
\end{frame}
\begin{frame}{Walsh Transform}
  Let the {\bf dual\ group} $\hat{\gftwo^n}$ be the group of all
  characters of $\gftwo^n$.
  \begin{itemize}
    \item[] \[ (\Chi\cdot\psi)(x)=\Chi(x)\psi(x),\ x\in\gftwo^n  \]
    \item[] \[\gftwo^n\cong\hat{\gftwo^n}\]
    \item[] \[\lambda\mapsto\Chi_\lambda\]
  \end{itemize}
\end{frame}

\begin{frame}{Walsh Transform}
  \begin{definition}\label{def:pBF}
    Let $f\in\BF_n$. Then $\hat{f}:\gftwo^n\rightarrow\{1,-1\}$ such that
    $\hat{f}(x)=(-1)^{f(x)}$ is a \textit{pseudo-Boolean function}
  \end{definition}
  \begin{example}
    Let $f=x_0+x_1$.
    \begin{table}
    \label{tab:truth-table}
    	\centering
      \begin{tabular}{|c|c||c|c|}
        \hline
        $x_0$&$x_1$&$f(x_0,x_1)$&$\hat{f}(x_0,x_1)$\\
        \hline
        0&0&0&1\\
        1&0&1&-1\\
        0&1&1&-1\\
        1&1&0&1\\
      	\hline
    	\end{tabular}
      \caption{Truth Table of $\hat{f}$}
    \end{table}
  \end{example}
\end{frame}

\begin{frame}{Walsh Transform}
  \begin{definition}\label{def:walsh}
    Let $f\in\BF_n$ and $\lambda\in\gftwo^n$. Then the {\em Walsh transform}
    of $f$ is defined by:
    \begin{equation}\label{eqn:walsh}
      \mathcal{W}_f(\lambda)=\sum_{x\in\gftwo^n}\hat{f}(x)\Chi_\lambda(x).
    \end{equation}
  \end{definition}
\end{frame}

\begin{frame}{Walsh Transform}
  \begin{lemma}
    The characters of $\gftwo^n$ belong to $\hat{\BF}_n
    =\{\hat{f}:f\in\BF_n\}$ and form an orthonormal basis of
    $\hat{\BF}_n\otimes\mathbb{R}$.
  \end{lemma}
  \begin{lemma}
    For $\hat{f}\in\hat{\BF}_n$,
  \begin{equation}\label{eqn:rewrite-pseudo}
  	\hat{f}(x)
      =\frac{1}{2^{n/2}}
        \sum_{\lambda\in\gftwo^n}c(\lambda)\Chi_\lambda(x)
  \end{equation}
  	where $c(\lambda)$ are given by
    \begin{equation}\label{eqn:clambda}
      c(\lambda)=\frac{1}{2^{n/2}}\mathcal{W}_f(\lambda)
    \end{equation}
  \end{lemma}
  Call the $c(\lambda)$'s {\bf Fourier coefficients}.
\end{frame}

\subsection{Bent Functions}
\begin{frame}{Rothaus' Definition and First Theorem}
  \begin{definition}\label{def:bent-function}
    If all of the Fourier coefficients of $\hat{f}$ are $\pm1$ then
    $f$ is a {\bf bent\ function}.
  \end{definition}
  \begin{theorem}\label{thm:deg-of-bent-function}
  	If $f$ is a bent function on $\gftwo^n$, then $n$ is even.
    Moreover, the degree of $f$ is at most $n/2$, except in the case $n=2$.
  \end{theorem}
\end{frame}

\begin{frame}{Properties of Bent Functions}
  \begin{enumerate}
      \pause
    \item[(R1) 1.] $wt(f)=2^{n-1}\pm2^{n/2-1}$
      \pause
    \item[(R2) 2.] perfectly non-linear
      \pause
    \item[(R3) 3.] $\sum_{x\in\gftwo^n}{f(x)+f(x+a)}=0 \ \ \forall a\in\gftwo^n$
  \end{enumerate}
\end{frame}

\section{Boolean Sequence}
\begin{frame}{Non-Linear Filtering}
  \setlength{\unitlength}{1mm}
  \begin{picture}(90,40)(-45,-25)
    %%filtering
    \put(0,15){\framebox(50,5){f}}
    \multiput(5,10)(10,0){2}{\line(0,4){5}}
    \multiput(35,10)(10,0){2}{\line(0,4){5}}
    \put(25,20){\vector(0,4){5}}
    %% State of the register
    \put(0,0){\framebox(10,10){$x_{n-1}$}}
    \put(10,0){\framebox(10,10){$x_{n-2}$}}
    \put(20,0){\framebox(10,10){$\dots$}}
    \put(30,0){\framebox(10,10){$x_{1}$}}
    \put(40,0){\framebox(10,10){$x_{0}$}}
    %% Taps
    \put(0,-13){\makebox(10,10){$q_1$}}
    \put(5,-8.5){\circle{9}}
    \put(10,-13){\makebox(10,10){$q_2$}}
    \put(15,-8.5){\circle{9}}
    \put(20,-13){\makebox(10,10){$\dots$}}
    \put(30,-13){\makebox(10,10){$q_{n-1}$}}
    \put(35,-8.5){\circle{9}}
    \put(40,-13){\makebox(10,10){$q_n$}}
    \put(45,-8.5){\circle{9}}
    %% Lines connecting Taps and the State of the Register
    \multiput(5,-0.2)(10,0){2}{\line(0,-6){4}}
    \multiput(35,-0.2)(10,0){2}{\line(0,-6){4}}
    %% Summer
    \put(-20,-25){\framebox(10,10){\Large $\sum$}}
    %% Lines from Taps to Summer
    \put(5,-12.8){\line(0,-6){3.5}}
    \put(15,-12.8){\line(0,-6){5.5}}
    \put(35,-12.8){\line(0,-6){8.5}}
    \put(45,-12.8){\line(0,-6){10.5}}
    \put(5,-16.3){\vector(-4,0){15}}
    \put(15,-18.3){\vector(-4,0){25}}
    \put(35,-21.3){\vector(-4,0){45}}
    \put(45,-23.3){\vector(-4,0){55}}
    %% Lines from Summer
    \put(-13,-15){\line(0,0){20}}
    \put(-13,5){\vector(4,0){13}}
    \put(-17,-15){\line(0,0){20}}
    \put(-17,5){\vector(-4,0){13}}
    %% Memory cell z
    \put(-40,0){\framebox(10,10){$z$}}
    %% Lines from memory cell to summer
    \put(-35,0){\line(0,-6){20}}
    \put(-35,-20){\vector(4,0){15}}
    %% div 2 mod 2
    \put(-13,5){\makebox(13,5){mod $2$}}
    \put(-29.5,5){\makebox(13,5){div $2$}}
  \end{picture}
\end{frame}
\begin{frame}{Boolean Sequence}
  \begin{definition}\label{def:lex-Bool-seq}
    Let $f\in\BF_n$ and $v_i\in\gftwo^n$ such that $v_i=B^{-1}(i)$ for
    $0\leq i<2^n$. Then,
    \begin{equation}
      seq(f)=(f(v_0),f(v_1),\cdots,f(v_{2^n-1}),f(v_0),\cdots)
    \end{equation}
    is a {\bf lexicographical\ Boolean\ sequence}.
  \end{definition}
  \begin{definition}
    Let $(a_n)$ be a sequence. If $T$ is the smallest integer such that
    $a_i=a_{i+T}$, then the {\bf minimal\ period} of $(a_n)$ is $T$.
  \end{definition}
  \begin{theorem}
    The lexicographical Boolean sequence of a Bent function has a period
    exactly $2^n$.
  \end{theorem}
\end{frame}

\begin{frame}{Boolean Sequence}
  \begin{definition}\label{2-adic-ex}
    Let $f\in\BF_n$ and $v_i\in\gftwo^n$ such that $v_i=B^{-1}(i)$ for
    $0\leq i<2^n$. Then,
    \begin{equation}
      \alpha_f=(f(v_0),f(v_0)+f(v_1)\cdot2,\cdots,\allowbreak
        f(v_0)+\cdots\allowbreak+f(v_i)\cdot2^i,\allowbreak\cdots)
    \end{equation}
    where $\alpha_f\in\zzz_2$ is called the {\bf 2-adic\ expansion} of $f$.
  \end{definition}
  
  \begin{lemma}
    The digit representation of $\alpha_f$ is $seq(f)$.
  \end{lemma}
\end{frame}

\subsection{Constructing}
\begin{frame}{Maiorana-McFarland Class Boolean Functions}
  \par A simple bent function construction is accomplished by the Boolean
  functions in the {\bf Maiorana-McFarland\ class}. This is the the set
  $\mathcal{M}$ which contains all Boolean functions on
  $\gftwo^n=\{(x,y):x,y\in\gftwo^{n/2}\}$, of the form:
    \[
    f(x,y)=x\cdot\pi(y)\oplus g(y)
    \]
  where $\pi$ is any permutation on $\gftwo^{n/2}$ and $g$ any Boolean
  function on $\gftwo^{n/2}$.\\
  \vspace{5mm} 
  \par All functions in the Maiorana-McFarland class of Boolean functions are
  bent.
\end{frame}
  
\begin{frame}
  \par Consider the subset of Maiorana-McFarland class Boolean functions where
  $g(y)=0$. $\bar{\pi}$ will be the function which specifies where each
  index moves to under the permutation $\pi$.
  \vspace{5mm}  
  \begin{theorem}
    $\log_2(\alpha_{x\cdot\pi(y)})=2^{n/2}+2^{\bar{\pi}(y_0)}$
  \end{theorem}
  \vspace{5mm}
  The 2-adic valuation of the Boolean sequence of the functions in this
  subset is entirely dependent on the permutation $\pi$.
\end{frame}

\end{document}
