\documentclass{beamer}
\usetheme{Rochester}
\usepackage{amsmath}
\usepackage{amssymb}
\usepackage{amsthm}
% Definitions
\def\qqq{\mathbb{Q}}
\def\rrr{\mathbb{R}}
\def\zzz{\mathbb{Z}}
\def\fff{\mathbb{F}}
\def\gftwo{\mathbb{F}_2}
\def\zzzp{\mathbb{Z}_p}
\def\zzzn{\mathbb{Z}_N}
\def\zg{\mathbb{Z}_g}
\def\nnn{\mathbb{N}}
\def\BF{\mathcal{BF}}
\def\xn{(x_n)}
\def\yn{(y_n)}
\def\zn{(z_n)}
\def\an{(a_n)}
\def\Chi{\raisebox{2pt}{$\chi$}}
% \def\qed{$\Box}
% \newcounter{padic}

\begin{document}
\title{Pseudorandom Sequences}
\author{Charles Celerier}
\frame{\titlepage}
\frame{\tableofcontents}

\section{Introduction}

\begin{frame}{Stream Ciphers}
  \begin{tabular}{c c c c}
              & 01001110010000110101010101010010 & = & NCUR\\
              \pause
     $\oplus$ & 00000101000010010001100100000010 & \\
              \hline 
              \pause
              & 01001011010010100100110001010000 & = & KJLP
  \end{tabular}
\end{frame}

\begin{frame}{Why use stream ciphers?}
  \begin{itemize}
    \item fast
    \item easy to implement with hardware
    \item plaintext length is not always known
    \item near one-time-pad security
  \end{itemize}
\end{frame}

\begin{frame}{My research}
  \begin{itemize}
    \item Boolean functions
    \item 2-adic integers
    \item pseudorandom sequences
    \item shift registers
  \end{itemize}
\end{frame}

\section{Boolean Functions}
\subsection{GF(2)}
\begin{frame}{$\gftwo$ or ``GF two''}
\begin{table}[h!]\label{tab:GF(2)}
	\centering
	\begin{tabular}{|c|c|}
		\hline
		XOR&AND\\
		\hline
		$0\oplus0:=0$&$0\cdot0:=0$\\
		$0\oplus1:=1$&$0\cdot1:=0$\\
		$1\oplus0:=1$&$1\cdot0:=0$\\
		$1\oplus1:=0$&$1\cdot1:=1$\\
		\hline
	\end{tabular}
	\caption{Binary Operations for $\gftwo$}
\end{table}
\end{frame}

\begin{frame}{$\gftwo^n$ or ``GF two to the n''}
\begin{example}
	Let $a,b\in\gftwo^3$ such that $a=(1,0,1)$ and $b=(0,1,1)$ then
	\begin{align*}
		a+b      &=(1\oplus0,0\oplus1,1\oplus1)=(1,1,0) \\
		a\cdot b &=1\cdot0\oplus0\cdot1\oplus1\cdot1=1
	\end{align*}
\end{example}

\begin{fact}
  $\gftwo^n$ is a vector space.
\end{fact}
\end{frame}

\begin{frame}{Properties of $x\in\gftwo^n$}
  \begin{definition}
  \label{def:Hamming}
  	Let $x,y\in\gftwo^n$. Then $wt:\gftwo^n\rightarrow\nnn\cup\{0\}$
    is defined by
  	\[
  	  wt(x):=\sum_{i=0}^{n-1}x_i
  	\]
  	and $d:\gftwo^n\times\gftwo^n\rightarrow\nnn\cup\{0\}$ is defined by
  	\[
  	  d(x,y):=w(x+y).
  	\]
  	Then $wt(x)$ is the {\em Hamming\ weight} of $x$ and $d(x,y)$ is the
  	{\em Hamming\ distance} between $x$ and $y$.
  \end{definition}
  \end{frame}
  
  \begin{frame}{Some examples}
  \begin{example}
  	Let $a,b,c\in\gftwo^5$ such that
  	\[
  	a=(0,1,1,0,1),\ b=(1,1,1,0,0),\ {\rm and}\ c=(0,0,1,1,0).
  	\]
  	Then,
  	\begin{center}
  		\begin{tabular}{c c}
  			$wt(a)=3$&$d(a,b)=2$\\
  			$wt(b)=3$&$d(a,c)=3$\\
  			$wt(c)=2$&$d(b,c)=3$.\\
  		\end{tabular}
  	\end{center}
  \end{example}
\end{frame}

\subsection{Boolean Functions}
\begin{frame}{Boolean functions in $\BF^n$}
  \begin{definition}
  \label{def:boolean-function}
    Any function $f$ defined such that 
    \begin{equation*}
      f:\gftwo^n\rightarrow\gftwo
    \end{equation*}
    is a {\em Boolean function}. The set of all Boolean functions on $n$
    variables will be denoted by $\BF_n$.
  \end{definition}
\end{frame}
\begin{frame}{An example}
  \begin{example}
    Let $f=x_0+x_1$.
    \begin{table}
    \label{tab:truth-table}
    	\centering
      \begin{tabular}{|c|c||c|}
        \hline
        $x_0$&$x_1$&$f(x_0,x_1)$\\
        \hline
        0&0&0\\
        1&0&1\\
        0&1&1\\
        1&1&0\\
      	\hline
    	\end{tabular}
    	\caption{Truth Table of $f$}
    \end{table}
  \end{example}
\end{frame}

\subsection{Discrete Fourier Transforms}
\begin{frame}{Discrete Fourier Transform}
  \begin{definition}
    $\hat{f}(x):=(-1)^{f(x)}$
  \end{definition}
  \begin{lemma}
  \begin{equation}\label{eqn:rewrite-pseudo}
  	\hat{f}(x)
      =\frac{1}{2^{n/2}}
        \sum_{\lambda\in\gftwo^n}c(\lambda)\Chi_\lambda(x)
  \end{equation}
  	where $c(\lambda)$, the {\em Fourier\ coefficients} of $\hat{f}(x)$ are
    given by
    \begin{equation}\label{eqn:clambda}
      c(\lambda)=\frac{1}{2^{n/2}}\mathcal{F}\hat{f}(\lambda).
        %=\frac{1}{2^{n/2}}\sum_{x\in\gftwo^n}(-1)^{f(x)+\lambda\cdot x}.
    \end{equation}
  \end{lemma}
\end{frame}

\end{document}
