\section{Bent Functions}
\par Bent functions were originally defined in a paper by Rothaus in the
Journal of Combinatorial Theory in 1976 \cite{art:r76}. These functions are
useful for cryptographic applications because they are {\em perfectly\ 
nonlinear} which makes them resistant to differential
attacks. The original defintion of the {\em bent\ function} is based on the
Fourier coefficients.

\subsection{The Discrete Fourier Transform}
\par The Fourier transform of a function has many applications in signals
analysis. The idea is to express a given signal as a function of frequency
to reveal the dominant frequencies in the signal. Today, digital signals
have taken over analog. This fact has made the discrete Fourier transform
much more applicaple to modern communication systems.

\par The discrete Fourier transforms (DFTs) of $n$-variable pseudo Boolean
functions are studied here. The following discussion on DFTs will follow
\cite{bk:t99}. To begin the discussion about DFTs, {\em characters} are
defined.

\begin{definition}\cite{bk:t99}
  A {\em character} $\Chi$ of a finite abelian group $G$ is a group
  homomorphism from $G$ into the multiplicative group of complex numbers.
\end{definition}

For the purposes of this paper, it should be clear that
$\Chi_\lambda(x):=(-1)^{\lambda\cdot x}$ where $\lambda,x\in\gftwo^n$ is a
{\em group\ character} of $\gftwo^n$. Define the {\em dual\ group}
$\hat{\gftwo^n}$ to be the set of all characters of $G$. The group operation
in $\hat{\gftwo^n}$ is pointwise multiplication of functions:
\[
\Chi\psi(x)=\Chi(x)\psi(x)
\]
This operation is closed under multiplication. The isomorphism
$\gftwo^n\cong\hat{\gftwo^n}$ is shown by constructing a one-to-one and onto
mapping $\Upsilon:\gftwo^n\rightarrow\hat{\gftwo^n}$ where
$\Upsilon(\lambda):=\Chi_\lambda$.

\begin{proof}
	Every character in $\hat{\gftwo^n}$ corresponds to an element of
  $\gftwo^n$. Thus, $\lvert\gftwo^n\rvert=\lvert\hat{\gftwo^n}\rvert$. If
  $\Upsilon$ is one-to-one, then
	it must be an isomorphism.\\
	Let $\Upsilon(\lambda_1)=\Upsilon(\lambda_2)$. Then $\forall x$
	\begin{align*}
		(-1)^{\lambda_1\cdot x}
      &=(-1)^{\lambda_2\cdot x}\\
		  &=(-1)^{(\lambda_1+\lambda_1+\lambda_2)\cdot x}\\
      &=(-1)^{\lambda_1\cdot x}(-1)^{(\lambda_1+\lambda_2)\cdot x}.
	\end{align*}
	Finally, $(\lambda_1+\lambda_2)\cdot x=0$ for all $x\in\gftwo^n$, which
  implies $\lambda_1+\lambda_2=0$. Therefore, $\lambda_1=\lambda_2$.
\end{proof}

\par Addition in $\gftwo^n$ corresponds to multiplication in
$\hat{\gftwo^n}$. Define
$\Chi_{\lambda_1}\Chi_{\lambda_2}(x)=(-1)^{(\lambda_1+\lambda_2)\cdot x}$.
Then,
\[
\Chi_{\lambda_1}(x)\Chi_{\lambda_2}(x)
  =(-1)^{\lambda_1\cdot x}(-1)^{\lambda_2\cdot x}
  =(-1)^{(\lambda_1+\lambda_2)\cdot x}
  =\Chi_{\lambda_1}\Chi_{\lambda_2}(x).
\]

\par For discrete Fourier transforms, it is useful to consider {\em pseudo\ 
Boolean\ functions}
which have codomain $\{-1,1\}$. 

\begin{definition}\label{def:pBF}
	Any function $\hat{f}:\gftwo^n\rightarrow\{1,-1\}$ such that
  $\hat{f}(x)=(-1)^{f(x)}$ for $f\in\BF$ is called a {\em pseudo Boolean
  function}
\end{definition}

\begin{definition}\label{def:DFT}
	The {\em discrete\ Fourier\ transform} or DFT of a Boolean function is
  defined by
	\begin{equation}\label{eqn:DFT}
    \mathcal{F}f(\lambda)
      =\sum_{x\in\gftwo^n}f(x)\Chi_\lambda(x)
	\end{equation}
\end{definition}

\begin{lemma}
  The characters of $\gftwo^n$ are functions in $\hat{\BF}_n
  =\{\hat{f}:f\in\BF_n$ and form an orthonormal basis of that set.
\end{lemma}
\begin{proof}
  The dimension of $\hat{\BF_n}$ is $2^n$, and there are $2^n$ characters of
  $\gftwo^n$.
  \begin{align*}
    \sum_{x\in\gftwo^n}{\Chi_{\lambda_i}(x)\cdot\Chi_{\lambda_j}(x)}
    &=\sum_{x\in\gftwo^n}
      {(-1)^{(\lambda_i+\lambda_j)\cdot x}}\\
    &=\begin{cases}
      0 & \text{if } i\not=j \\
      2^n & \text{if } i=j.
    \end{cases}
  \end{align*}
  Therefore, the characters of $\gftwo^n$ form an orthonormal basis of
  $\hat{\BF}_n$.
\end{proof}

\par Every pseudo Boolean function can be written as a linear combination of
the characters of $\gftwo^n$. The coefficients in these linear combinations
reveal important properties of the functions. Rothaus rewrote the pseudo
Boolean function as a linear combination as follows \cite{art:r76}. 

\begin{lemma}
\begin{equation}\label{eqn:rewrite-pseudo}
	\hat{f}(x)
    =\frac{1}{2^{n/2}}
      \sum_{\lambda\in\gftwo^n}c(\lambda)\Chi_\lambda(x)
\end{equation}
	where $c(\lambda)$, the {\em Fourier\ coefficients} of $\hat{f}(x)$ are
  given by
	\[
  c(\lambda)=\frac{1}{2^{n/2}}\mathcal{F}\hat{f}(\lambda).
    %=\frac{1}{2^{n/2}}\sum_{x\in\gftwo^n}(-1)^{f(x)+\lambda\cdot x}.
	\]
\end{lemma}

\par As observed by Rothaus \cite{art:r76}, $2^{n/2}c(\lambda)$ is the
number of zeros minus the number of ones of the function
$f(x)+\lambda\cdot x$. The Hamming weight of f is easily determined using
the zero Fourier coefficient $c(0)$:
\begin{align*}
	c(0)&=\frac{1}{2^{n/2}}\sum_{x\in\gftwo^n}(-1)^{f(x)}\\
	&=\frac{1}{2^{n/2}}\big((2^n-wt(f))-wt(f)\big)
\end{align*}
\begin{equation}
  \Rightarrow wt(f)=2^{n-1}-2^{n/2-1}c(0).
\end{equation}

\begin{definition}\label{def:walsh}
  Let $f\in\BF_n$ and $\lambda\in\gftwo^n$. Then the {\em Walsh transform}
  of $f$ is defined by:
  \begin{equation}\label{eqn:walsh}
    \mathcal{W}_f(\lambda)=\mathcal{F}\hat{f}(\lambda).
  \end{equation}
\end{definition}

\par Clearly $\mathcal{W}_f(\lambda)=\frac{1}{2^{n/2}}c(\lambda)$. For large
$|\mathcal{W}_f(\lambda)|$, $f$ is close to an affine function in $\BF_n$.

\begin{example}
  There are a two cases that should be clear to the reader.
  \begin{enumerate}[1.]
    \item Let $f(x)=\lambda\cdot x$. Then $\mathcal{W}_f(\lambda)=2^n$.
    \item Let $f(x)\not=\lambda\cdot x \ \ \forall x$. Then $f(x)=
      \lambda\cdot x+1 \ \ \forall x$. So, $\mathcal{W}_f(\lambda)=-2^n$
  \end{enumerate}
  In both cases, $f$ is an affine function.
\end{example}

\subsection{Bent Functions}
\begin{definition}\label{def:bent-function}
  If all of the Fourier coefficients of $\hat{f}$ are $\pm1$ then
  $f$ is a {\em bent\ function}.
\end{definition}

\par Here is Rothaus's first theorem about bent functions.

\begin{theorem}\label{thm:deg-of-bent-function}
	If $f$ is a bent function on $\gftwo^n$, then $n$ is even, $n=2k$;
	the degree of $f$ is at most $k$, except in the case $k=1$.
\end{theorem}
\begin{proof}
  \par $c(\lambda)=\pm1$. This implies $2^{n/2}c(\lambda)$ is an
  integer. Therefore $n$ must be even.
  \par Let $n=2k$ with $k>1$, and let $r>k$. Consider the polynomial
  $f(x_1,x_2,\allowbreak\dots,\allowbreak x_r,\allowbreak 0,0,\dots,0)=
  \allowbreak g(x_1,x_2,\dots,x_r)$ (up to this point all numbering has
  started at 0; it is more convenient in this proof to begin numbering at
  1). Then by Equation (\ref{eqn:rewrite-pseudo}),
  \[
  \hat{g}(x)=\frac{1}{2^{r/2}}\sum_{\lambda_1,\lambda_2,\dots,\lambda_r=0,1}
    b(\lambda_1,\dots,\lambda_r)\Chi_{(\lambda_1,\ldots,\lambda_r)}(x)
  \]
  and
  \[
	\hat{f}(x,0)=\frac{1}{2^{n/2}}
    \sum_{\lambda_1,\lambda_2,\dots,\lambda_n=0,1}
    c(\lambda_1,\dots,\lambda_n)\Chi_{(\lambda_1,\ldots,\lambda_n)}(x,0).
  \]
  Because $f(x,0)=g(x)$ and the uniqueness of the Fourier expansion, $b$ and
  $c$ are related such that
  \[
  b(\lambda_1,\dots,\lambda_r)
    =\frac{1}{2^{(n-r)/2}}\sum_{\lambda_{r+1},\dots,\lambda_n=0,1}
    c(\lambda_1,\dots,\lambda_r,\lambda_{r+1},\dots,\lambda_n).
  \]
  \par Then,
  \begin{align*}
  wt(f(x,0))&=wt(g(x))\\
    &=2^{r-1}-2^{r/2-1}b(0)\\
    &=2^{r-1}-2^{r-n/2-1}\sum_{\lambda_{r+1},\dots,\lambda_n=0,1}
      c(0,\dots,0,\lambda_{r+1},\dots,\lambda_n).
  \end{align*}
  \par There are $2^{n-r}$ summands in
  $\sum{c(0,\dots,0,\lambda_{r+1},\dots,\lambda_n)}.$ Since $f$ is bent,
  $c(\lambda)=\pm1$. By rewriting $1=-1+2$,
  \begin{align*}
    \sum{c(0,\dots,0,\lambda_{r+1},\dots,\lambda_n)}
      &=-2^{n-r}+2wt(c(0,\dots,0,\lambda_{r+1},\dots,\lambda_n))\\
      &=2\big(wt(c(0,\dots,0,\lambda_{r+1},\dots,\lambda_n))-2^{n-r-1}\big)
  \end{align*}
  \par Thus, $wt(g(x))$ is even. This implies that $g(x)$ is the sum of an
  even number of atomic Boolean functions. Therefore the coefficient of
  $x_1x_2\cdots x_r$ in the polynomial representing $g(x)$ must be 0. This
  is true for every $r>k$, so the degree of $f$ must not be greater than
  $k$.
\end{proof}

% Strict Avalanche Criterion
% directional derivatives
% equivalent definition of a bent function

\subsection{Constructions of Bent Functions}
\par A simple bent function construction is accomplished by the Boolean
functions in the {\em Maiorana-McFarland\ class}. This is the the set
$\mathcal{M}$ which contains all Boolean function on
$\gftwo^n=\{(x,y):x,y\in\gftwo^{n/2}\}$, of the form:
  \[
  f(x,y)=x\cdot\pi(y)\oplus g(y)
  \]
where $\pi$ is any permutation on $\gftwo^{n/2}$ and $g$ any Boolean
function on $\gftwo^{n/2}$.

\par All functions in the Maiorana-McFarland class of Boolean functions are
bent.
