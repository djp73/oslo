\section{Bent Functions}
\par Bent functions were originally defined in a paper by Rothaus in the Journal
of Combinatorial Theory in 1976 \cite{art:r76}. These functions are useful for
cryptographic applications because they are {\em perfectly\ nonlinear} which
makes them resistant to differential attacks. The original defintion of the
{\em bent\ function} is presented.

\par First, the discrete Fourier transform should be understood.

% Insert discussion about DFTs

% Insert discussion about Hadammard matrices

\par If $f(x)\in\BF_n$, so $(-1)^{f(x)}$ is well defined, then by the
theory of discrete fourier transforms
\begin{equation}\label{eqn:DFT}
  (-1)^{f(x)}=\frac{1}{2^{n/2}}\sum_{\lambda\in\gftwo^n}c(\lambda)(-1)^{\lambda\cdot x}
\end{equation}
Where the $c(\lambda)$, the {\em Fourier\ coefficients} of $(-1)^{f(x)}$ are given by
\[
  c(\lambda)=\frac{1}{2^{n/2}}\sum_{x\in\gftwo^n}(-1)^{f(x)}(-1)^{\lambda\cdot x}.
\]
As observed by Rothaus, $2^{n/2}c(\lambda)$ is the number of zeros minus the number
of ones of the function $f(x)+\lambda\cdot x$. It is clear that when there are more
zeros than ones $c(\lambda)>0$ and when there are more ones than zeros $c(\lambda)<0$.
Counting the number of zeros in $f(x)$ is also easily seen. Consider the zero Fourier
coefficient $c(0)$:
\begin{align*}
	c(0)&=\frac{1}{2^{n/2}}\sum_{x\in\gftwo^n}(-1)^{f(x)}\\
	&=\frac{1}{2^{n/2}}{\rm \ number\ of\ zeros\ of\ }f(x)-{\rm \ number\ of \ ones\ of \ }f(x).
\end{align*}
Since $2^n=\ $ number of zeros of $f(x)+$ number of ones of $f(x)$,
\begin{align*}
	{\rm number\ of\ zeros\ of\ }f(x)&=\frac{2^{n/2}c(0)+2^n}{2}\\
	                                 &=2^{n/2-1}c(0)+2^{n-1}.
\end{align*}

\begin{definition}\label{def:bent-function}
  If all of the Fourier coefficients of $(-1)^{f(x)}$ are $\pm1$ then
  $f(x)$ is a {\em bent\ function}.
\end{definition}

\par There are two immediate observations for any bent function $f(x)$ on $\gftwo^n$.
First, $n$ must be even because $2^{n/2}c(\lambda)$ is an integer. Second, the Hamming
weight of $f(x)$ equals $2^{n-1}\pm2^{n/2-1}c(0)$. There also is a bound on the degree
of the polynomial of $f(x)$ proved by Rothaus. The theorem and proof are presented here.

\begin{theorem}\label{thm:deg-of-bent-function}
	If $f(x)$ is a bent function on $\gftwo^n$, then $n$ is even, $n=2k$;
	the degree of $f(x)$ is at most $k$, except in the case $k=1$.
\end{theorem}
\begin{proof}
\par That $n$ is even has already been observed.
\par Let $n=2k$ with $k>1$, and let $r>k$. Consider the polynomial
$f(x_1,x_2,\dots,x_r,0,0,\dots,0)=g(x_1,x_2,\dots,x_r)$. Then by equation \ref{eqn:DFT},
\[
	(-1)^{g(x)}=\frac{1}{2^{r/2}}\sum_{\lambda_1,\lambda_2,\dots,\lambda_r=0,1}b(\lambda_1,\dots,\lambda_r)(-1)^{\lambda_1x_1+\dots+\lambda_rx_r}
\]
and
\[
	(-1)^{f(x)}=\frac{1}{2^{n/2}}\sum_{\lambda_1,\lambda_2,\dots,\lambda_n=0,1}c(\lambda_1,\dots,\lambda_n)(-1)^{\lambda_1x_1+\dots+\lambda_rx_r}.
\]
Because $f(x)=g(x)$ and the uniqueness of the Fourier expansion, $b$ and $c$ are related such that
\[
b(\lambda_1,\dots,\lambda_r)=\frac{1}{2^{(n-r)/2}}\sum_{\lambda_{r+1},\dots,\lamda_n=0,1}c(\lambda_1,\dots,\lamda_r,\lamda_{r+1},\dots,\lamda_n).
\]
Each $b(\lambda_1,\dots,\lambda_r)$ is a sum of $c(\lambda_1,\dots,\lambda_n)$'s in this way.
\par The number of zeros of $g(x_1,\dots,x_r)=f(x_1,\dots,x_r,0,0,\dots,0)$ equals
$2^{r-1}+

% Talk about some different properties associated with the bent functions
%  -weight is 2^(n/2)\pm2^(n\2-1)

\par A simple bent function construction is accomplished by the Boolean functions
in the {\em Maiorana-McFarland\ class}. This is the the set $\mathcal{M}$ which
contains all Boolean function on $\gftwo^n=\{(x,y):x,y\in\gftwo^n\}$, of the form:
  \[
  f(x,y)=x\cdot\pi(y)\oplus g(y)
  \]
where $\pi$ is any permutation on $\gftwo^{n/2}$ and $g$ any Boolean function on
$\gftwo^{n/2}$.

\par All function in the Maiorana-McFarland class of Boolean functions are bent.
