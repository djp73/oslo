\section{Bent Functions}
\par A bent function is a special type of boolean function. It is sometimes
called a perfectly non-linear function. Before introducing a bent function
a boolean function is defined.

% Definition of a Boolean Function
\begin{definition}
\label{def:boolean-function}
  Any function $BF$ with the property
  \begin{equation}
    BF:\fff_2^n\rightarrow\fff_2
  \end{equation}
  is a {\em boolean function}.
\end{definition}

\par Oftentimes the codomain of a boolean function represents values of true
and false for use in logic circuits. The number of boolean functions increases
extremely rapidly as the number of variables increases.

\begin{equation}
  \lVert\{BF:\fff_2^n\rightarrow\fff_2\}\rVert = 2^{2^n}
\end{equation}

\par As observed by Carlet, consider the set of all boolean functions on 7 variables,
and say that one nanosecond is spent at each function to identify the function and
note some properties about it. If we visited every boolean function this way, it
would take 100 billions times the age of the universe to complete the search.
For eight variables, there are more boolean functions than there are atoms in the
universe.

\par As always, it is best to view an example. Consider the following boolean function
$f$.
\\
\\
\begin{tabular}{|c|c|c|c|c|}
  \hline
  $x_4$&$x_3$&$x_2$&$x_1$&$f(x_4,x_3,x_2,x_1)$\\
  \hline
  0&0&0&0&0\\
  0&0&0&1&1\\
  0&0&1&0&1\\
  0&0&1&1&0\\
  0&1&0&0&1\\
  0&1&0&1&0\\
  0&1&1&0&1\\
  0&1&1&1&0\\
  1&0&0&0&0\\
  1&0&0&1&0\\
  1&0&1&0&1\\
  1&0&1&1&0\\
  1&1&0&0&0\\
  1&1&0&1&0\\
  1&1&1&0&1\\
  1&1&1&1&1\\
  \hline
\end{tabular}
\\
\\
\par The function $f$ can be written as a sum of many different functions.
In particular, it is convenient to use {\em standard boolean functions} which
equal $1$ at exactly one point in $\fff_2^n$. Then every boolean function can
be written as a sum of $w(f)$ standard boolean functions. For the above example,
we have $f=f_2+f_3+f_5+f_7+f_{11}+f_{15}+f_{16}$.
\\
\\
\begin{tabular}{|c|c|c|c|c|c|c|c|c|c|c|c|}
  \hline
  $x_4$&$x_3$&$x_2$&$x_1$&$f$&$f_2$&$f_3$&$f_5$&$f_7$&$f_{11}$&$f_{15}$&$f_{16}$\\
  \hline
  0&0&0&0&0&0&0&0&0&0&0&0\\
  0&0&0&1&1&1&0&0&0&0&0&0\\
  0&0&1&0&1&0&1&0&0&0&0&0\\
  0&0&1&1&0&0&0&0&0&0&0&0\\
  0&1&0&0&1&0&0&1&0&0&0&0\\
  0&1&0&1&0&0&0&0&0&0&0&0\\
  0&1&1&0&1&0&0&0&1&0&0&0\\
  0&1&1&1&0&0&0&0&0&0&0&0\\
  1&0&0&0&0&0&0&0&0&0&0&0\\
  1&0&0&1&0&0&0&0&0&0&0&0\\
  1&0&1&0&1&0&0&0&0&1&0&0\\
  1&0&1&1&0&0&0&0&0&0&0&0\\
  1&1&0&0&0&0&0&0&0&0&0&0\\
  1&1&0&1&0&0&0&0&0&0&0&0\\
  1&1&1&0&1&0&0&0&0&0&1&0\\
  1&1&1&1&1&0&0&0&0&0&0&1\\
  \hline
\end{tabular}
\\
\\
\par This makes it easy to construct the original function $f$ because
the standard boolean functions are well-known.

\begin{align*}
  f_2   &=(1\oplus x_4)(1\oplus x_3)(1\oplus x_2)x_1\\
        &=x_1 \oplus x_2x_1 \oplus x_3x_1 \oplus x_3x_2x_1 \oplus x_4x_1 \oplus x_4x_2x_1
        \oplus x_4x_3x_1 \oplus x_4x_3x_2x_1\\
  f_3   &=(1\oplus x_4)(1\oplus x_3)x_2(1\oplus x_1)\\
  f_5   &=(1\oplus x_4)x_3(1\oplus x_2)(1\oplus x_1)\\
  f_7   &=(1\oplus x_4)x_3x_2(1\oplus x_1)\\
  f_{11}&=x_4(1\oplus x_3)x_2(1\oplus x_1)\\
  f_{15}&=x_4x_3x_2(1\oplus x_1)\\
  f_{16}&=x_4x_3x_2x_1\\
\end{align*}

\par Writing out all of the standard boolean functions on $n$ variables, it is clear that
the number of terms in $f_i$ equals $2^{2^{n}}-i+1$. Moreover, as $i$ increases, the terms that
drop off are those which correspond to the binary numbers less than $i-1$. To see this, associate
each term with a binary number in the following way:

\begin{align*}
  x_1&=2^0 , x_2=2^1 , x_3=2^2 , x_4=2^3 \\
  \\
  1&=0\\
  x_1&=1\\
  x_2&=2\\
  x_2x_1&=3\\
  &\vdots\\
  x_4x_3x_2x_1&=15\\
\end{align*}

\par Recognizing this association with the number of terms and which ones drop off as
$i$ increases for the standard boolean functions the following formula should now be clear.

\begin{equation}
  f(x_n,\cdots,x_1)=\bigoplus_{u\in\fff_2^n}a_u\lgroup\prod_{j=1}^nx_j^{u_j}\rgroup.
\end{equation}
  

%\section{The Ring of $N$-adic Integers}
%\par The notation used in the definition of the $N$-adic numbers will follow
%the same notation used by Borevich and Shafarevich in Chapter 1 of
%{\bf Number Theory}.
%
%\par In this section, the set of $N$-adic integers is shown to be a
%commutative ring with an identity.
%  
%% Definition of N-adic integers
%\begin{definition}
%\label{def:N-adic}
%  Let $N$ be an integer. Then the infinite integer sequence $\xn$
%  determines a {\em $N$-adic integer} $\alpha$, or $\xn \rightarrow \alpha$, if
%\begin{equation} \label{eq:seq}
%  x_{i+1} \equiv x_i \pmod{N^{i+1}} \ \ \ \forall i \geq 0.
%\end{equation}
%  Two sequences $\xn$ and $\{x_n'\}$ determine the same $N$-adic integer
%  if 
%\begin{equation} \label{eq:equiv}
%  x_i \equiv x_i' \pmod{N^{i+1}}\ \ \ \forall i \geq 0.
%\end{equation}
%  The {\em set of all $N$-adic integers} will be denoted by $\zzzn$.
%\end{definition}
%
%\par Ordinary integers will be called {\em rational integers} and each
%rational integer $x$ is associated with a $N$-adic integer, determined
%by the sequence \{$x,\ x, \ \dots, \ x,\ \dots$\}.
%	
%% example of equivalent sequences in Zp
%\begin{example} \label{ex:equiv-seq}
%  Let $\xn \rightarrow \alpha \in \zzz_3$. Then the first 5 terms of
%  $\xn$ may look something like:
%  \begin{align*}
%    \xn = \{&1 \ , \ 1+2\cdot3 \ , \ 1+2\cdot3+1\cdot3^2 \ , \\
%    &1+2\cdot3+1\cdot3^2 \ , \ 1+2\cdot3+1\cdot3^2+1\cdot3^4 \ , \ \dots\} \\
%        = \{&1,7,16,16,97,\dots\}
%  \end{align*}
%  Then equivalent sequences to $\xn$ could begin differently for the
%  first few terms:
%  \begin{align*}
%    \yn &= \{4,25,16,178,583,\dots\} \\
%    \zn &= \{-2,-47,232,-308,97,\dots\}
%  \end{align*}
%  The sequences for $\yn$ and $\zn$ satisfy equation (\ref{eq:seq})
%  for the first 5 terms, so they could be $N$-adic integers up to this
%  point. Also, both are equivalent to $\xn$ according to the
%  equivalence defined in equation (\ref{eq:equiv}).
%  \begin{align*}
%    &1 \equiv 4 \equiv 2 \pmod 3 \\
%    &7 \equiv 25 \equiv -47 \pmod{3^2} \\
