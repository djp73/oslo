\section{Bent Functions}
\par A bent function is a special type of boolean function. It is sometimes
called a perfectly non-linear function. Before introducing a bent function
a boolean function is defined.

% Definition of a Boolean Function
\begin{definition}
\label{def:boolean-function}
  Any function $\mathcal{BF}$ with the property
  \begin{equation}
    \mathcal{BF}:F\fff_2^n\rightarrow\fff_2
  \end{equation}
  is a {\em boolean function}.
\end{definition}

\par Oftentimes the codomain of a boolean function represents values of true
and false for use in logic circuits. The number of boolean functions increases
extremely rapidly as the number of variables increases.

\begin{equation}
  \lvert\{\mathcal{BF}:\fff_2^n\rightarrow\fff_2\}\rvert = 2^{2^n}
\end{equation}

\par As observed by Carlet, consider the set of all boolean functions on 7 variables,
and say that one nanosecond is spent at each function to identify the function and
note some properties about it. If we visited every boolean function this way, it
would take 100 billions times the age of the universe to complete the search.
For eight variables, there are more boolean functions than there are atoms in the
universe.

\par As always, it is best to view an example. Consider the following boolean function
$f$.
\\
\\
\begin{tabular}{|c|c|c|c|c|}
  \hline
  $x_3$&$x_2$&$x_1$&$x_0$&$f(x_3,x_2,x_1,x_0)$\\
  \hline
  0&0&0&0&0\\
  0&0&0&1&1\\
  0&0&1&0&1\\
  0&0&1&1&0\\
  0&1&0&0&1\\
  0&1&0&1&0\\
  0&1&1&0&1\\
  0&1&1&1&0\\
  1&0&0&0&0\\
  1&0&0&1&0\\
  1&0&1&0&1\\
  1&0&1&1&0\\
  1&1&0&0&0\\
  1&1&0&1&0\\
  1&1&1&0&1\\
  1&1&1&1&1\\
  \hline
\end{tabular}
\\
\\
\par The function $f$ can be written as a sum of many different functions.
In particular, it is convenient to use {\em atomic boolean functions}.

\begin{definition}
  An {\em atomic boolean function} is a boolean function that equals 1 for exactly
  one input.
\end{definition}

Then every boolean function can be written as a sum of $w(f)$ standard boolean
functions. For the above example, we have $f=f_1+f_2+f_4+f_6+f_{10}+f_{14}+f_{15}$.
\\
\\
\begin{tabular}{|c|c|c|c|c|c|c|c|c|c|c|c|}
  \hline
  $x_3$&$x_2$&$x_1$&$x_0$&$f$&$f_1$&$f_2$&$f_4$&$f_6$&$f_{10}$&$f_{14}$&$f_{15}$\\
  \hline
  0&0&0&0&0&0&0&0&0&0&0&0\\
  0&0&0&1&1&1&0&0&0&0&0&0\\
  0&0&1&0&1&0&1&0&0&0&0&0\\
  0&0&1&1&0&0&0&0&0&0&0&0\\
  0&1&0&0&1&0&0&1&0&0&0&0\\
  0&1&0&1&0&0&0&0&0&0&0&0\\
  0&1&1&0&1&0&0&0&1&0&0&0\\
  0&1&1&1&0&0&0&0&0&0&0&0\\
  1&0&0&0&0&0&0&0&0&0&0&0\\
  1&0&0&1&0&0&0&0&0&0&0&0\\
  1&0&1&0&1&0&0&0&0&1&0&0\\
  1&0&1&1&0&0&0&0&0&0&0&0\\
  1&1&0&0&0&0&0&0&0&0&0&0\\
  1&1&0&1&0&0&0&0&0&0&0&0\\
  1&1&1&0&1&0&0&0&0&0&1&0\\
  1&1&1&1&1&0&0&0&0&0&0&1\\
  \hline
\end{tabular}
\\
\\
\par This makes it easy to construct the original function $f$ because
the standard boolean functions are well-known.

\begin{align*}
  f_1   &=(1\oplus x_3)(1\oplus x_2)(1\oplus x_1)x_0\\
        &=x_0 \oplus x_1x_0 \oplus x_2x_0 \oplus x_2x_1x_0 \oplus x_3x_0 \oplus x_3x_1x_0
        \oplus x_3x_2x_0 \oplus x_3x_2x_1x_0\\
  f_2   &=(1\oplus x_3)(1\oplus x_2)x_1(1\oplus x_0)\\
  f_4   &=(1\oplus x_3)x_2(1\oplus x_1)(1\oplus x_0)\\
  f_6   &=(1\oplus x_3)x_2x_1(1\oplus x_0)\\
  f_{10}&=x_3(1\oplus x_2)x_1(1\oplus x_0)\\
  f_{14}&=x_3x_2x_1(1\oplus x_0)\\
  f_{15}&=x_3x_2x_1x_0\\
\end{align*}

\par Rather than continuing the computation and finding the polynomial for the boolean function
$f$, an algorithm will be introduced for computing the polynomial coefficients for any boolean
function.

\begin{theorem}
  For a boolean function $f$ of the form $f(x)=\bigoplus_{I\in\mathcal{P}(N)}a_Ix^I$, 
  \begin{equation}
    \forall I \in \mathcal{P}(N) \ \ a_I=\bigoplus_{supp(y)\subseteq I}f(y)
  \end{equation}
\end{theorem}


