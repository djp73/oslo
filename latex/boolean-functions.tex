\section{Boolean Functions}
\par This section will establish the definition of a {\em Boolean function} and introduce
a few different tools used to measure properties of these functions. First, the
finite field with two elements, denoted $\gftwo$, is defined. The definitions and
notations will follow those found in \cite{bk:cs09}.

% Definition of the finited field GF(2)
\par The two element field $(\gftwo,\oplus,\cdot)$ is the set $\{0,1\}$ with defined binary
operations $\oplus$ and $\cdot$, also commonly referred to as the {\em XOR} and
{\em AND} operators, respectively.
\begin{figure}[h!]\label{fig:GF(2)}
	\centering
	\begin{tabular}{|c|c|}
		\hline
		XOR&AND\\
		\hline
		$0\oplus0:=0$&$0\cdot0:=0$\\
		$0\oplus1:=1$&$0\cdot1:=0$\\
		$1\oplus0:=1$&$1\cdot0:=0$\\
		$1\oplus1:=0$&$1\cdot1:=1$\\
		\hline
	\end{tabular}
	\caption{Binary Operations for $\gftwo$}
\end{figure}
\par It should be clear that $(\gftwo,\oplus,\cdot)$ is a commutative ring with an
identity. Additionally, the only non-zero element $1$ is it's own inverse. Therefore
$(\gftwo,\oplus,\cdot)$ is a finite field. The $n$-dimensional vector space over $\gftwo$
will be denoted by $\gftwo^n$, with the usual inner product. Components of these vectors,
the individual 1s and 0s, will be known as {\em bits}. For two vectors $x,y\in\gftwo^n$
where $x=(x_0,\cdots,x_{n-1})$ and $y=(y_0,\cdots,y_{n-1})$, the scalar product in
$\gftwo^n$ will be defined as $x\cdot y\equiv x_0\cdot y_0 \oplus \cdots \oplus x_{n-1}\cdot y_{n-1}$.

\begin{example}
	Let $a,b\in\gftwo^3$ such that $a=(1,0,1)$ and $b=(0,1,1)$ then
	\begin{align*}
		a+b      &=(1\oplus0,0\oplus1,1\oplus1)=(1,1,0) \\
		a\cdot b &=1\cdot0\oplus0\cdot1\oplus1\cdot1=1
	\end{align*}
\end{example}

\par Each vector in $\gftwo^n$ can be uniquely represented by an integer between
$0$ and $2^{n-1}$. The binary representation function $B$ is one-to-one.
\begin{equation}
	B:\gftwo^n\rightarrow\nnn\cup\{0\}\ {\rm such\ that }\ B(u)\equiv \sum_{i=0}^{n-1}u_i\cdot2^i.
\end{equation}

\par This definition of $B$ has created the convention where the {\em low bit} or
{\em least significant bit} appears on the left and the {\em high bit} or {\em most significant bit}
appears on the right. There is are two important functions which will be used extensively in this
section. One counts the number of 1s in each vector and the other locates where those ones occur.

\begin{definition}
\label{def:Hamming}
	Let $u,v\in\gftwo^n$. Then $wt:\gftwo^n\rightarrow\nnn\cup\{0\}$ is defined by
	\[
	  wt(u):=\sum_{i=0}^{n-1}u_i
	\]
	and $d:\gftwo^n\times\gftwo^n\rightarrow\nnn\cup\{0\}$ is defined by
	\[
	  d(u,v):=w(u+v).
	\]
	Then $wt(u)$ is the {\em Hamming\ weight} of $u$ and $d(u,v)$ is the
	{\em Hamming\ distance} between $u$ and $v$.
\end{definition}

\begin{remark}
	The Hamming weight of a vector $u\in\gftwo^n$ is the number of 1s in the vector, and the
	Hamming distance between two vectors $u,v\in\gftwo^n$ is the number of bit differences
	between the two vectors.
\end{remark}

\begin{definition}
\label{def:support}
	Let $u\in\gftwo^n$. Then $supp:\gftwo^n\rightarrow\mathcal{P}(\nnn\cup\{0\})$ is defined by
	\[
		supp(u):=\{i\in\nnn\cup\{0\}:u_i\not=0\}
	\]
\end{definition}

\begin{example}
	Let $a,b,c\in\gftwo^5$ such that
	\[
	a=(0,1,1,0,1),\ b=(1,1,1,0,0),\ {\rm and}\ c=(0,0,1,1,0).
	\]
	Then,
	\begin{center}
		\begin{tabular}{c c c}
			$wt(a)=3$&$supp(a)=\{1,2,4\}$&$d(a,b)=2$\\
			$wt(b)=3$&$supp(b)=\{0,1,2\}$&$d(a,c)=3$\\
			$wt(c)=2$&$ supp(c)=\{2,3\}$ &$d(b,c)=3$.\\
		\end{tabular}
	\end{center}
\end{example}

\par If we define the vectors $v_i\in\gftwo^n$ such that $v_i=B^{-1}(i)$, then the sequence
$(v_0,v_1,\cdots,v_{2^n-1})$ is said to be in {\em lexicographical order}. Using the convention
of lexicographical ordering and the inner product in $\gftwo^n$, a $2^n\times 2^n$
matrix can be written which captures all of the possible inner products of two vectors in
$\gftwo^n$. This is a matrix where $u_i\cdot v_j$ appears in the $i$th row and $j$th column.
This particular matrix turns out to be a {\em Hadamard matrix} which will be studied later.

\par There is an interesting orthogonality property in the vectorspace $\gftwo^n$ known as the
{\em orthogonality principle} that every non-zero vector in $\gftwo^n$ is orthogonal to
exactly half of the vectors in the vectorspace.

\begin{theorem}
	\label{thm:orthogonality-principle}
	Let $u\in\gftwo^n$. Then
	\begin{align*}
		\sum_{v\in\gftwo^n}(-1)^{u\cdot v} &= 2^n {\rm \ for\ } u=0 \\
		                                   &= 0 {\rm \ otherwise.}
  \end{align*}
\end{theorem}

\begin{proof}
	Let $u=0\in\gftwo^n$. Then $\forall\ v\in\gftwo^n$, $u\cdot v=0$, so
	$(-1)^{u\cdot v}=1$. Therefore $\sum_{v\in\gftwo^n}(-1)^{u\cdot v}=\lvert\gftwo^n\rvert=2^n$. \\

	Let $u\in\gftwo^n$ where $u\not=0$. Assume the $i$th bit of $u$ is non-zero and
	define $e_i\in\gftwo^n$ as a vector with all zero bits except for the $i$th bit which is $1$. Then
	\begin{align*}
		\sum_{v\in\gftwo^n}(-1)^{u\cdot v} &= \sum_{v\in\gftwo^n}(-1)^{u\cdot (v+e_i)} \\
		                                   &= \sum_{v\in\gftwo^n}(-1)^{u\cdot v}(-1)^{u\cdot e_i} \\
																			 &= -\sum_{v\in\gftwo^n}(-1)^{u\cdot v}.
  \end{align*}
	Therefore, $\sum_{v\in\gftwo^n}(-1)^{u\cdot v}=-\sum_{v\in\gftwo^n}(-1)^{u\cdot v}$, which implies
	$\sum_{v\in\gftwo^n}(-1)^{u\cdot v}=0$.
\end{proof}

\par When introduced to a new vector field, it is natural to begin looking at functions in
that field. The particular function of interest here will be what is known as a {\em Boolean function}.

% Definition of a Boolean Function
\begin{definition}
\label{def:boolean-function}
  Any function $f$ defined such that 
  \begin{equation*}
    f:\gftwo^n\rightarrow\gftwo
  \end{equation*}
  is a {\em Boolean function}.\\
	\\
	The set of all Boolean function on $n$ variables will be denoted by $\mathcal{BF}_n$
\end{definition}

\par Boolean functions have been studied extensively, and there are various properties that
are used to characterize them. The number of Boolean functions increases extremely rapidly
as the number of variables increases.

\begin{equation}
  \lvert\{\mathcal{BF}_n:\fff_2^n\rightarrow\fff_2\}\rvert = 2^{2^n}
\end{equation}

\par As observed by Carlet, consider the set of all Boolean functions on 7 variables,
and say that one nanosecond is spent at each function to identify the function and
note some properties about it. If we visited every Boolean function this way, it
would take 100 billions times the age of the universe to complete the search.
For eight variables, there are more Boolean functions than there are atoms in the
universe (~$10^{80}$).

\par Here is an example of a Boolean function presented in a {\em truth\ table}.
\begin{table}
\label{tab:truth-table}
	\centering
  \begin{tabular}{|c|c|c|c|c|}
    \hline
    $x_0$&$x_1$&$x_2$&$x_3$&$f(x_0,x_1,x_2,x_3)$\\
    \hline
    0&0&0&0&0\\
    1&0&0&0&1\\
    0&1&0&0&1\\
    1&1&0&0&0\\
    0&0&1&0&1\\
    1&0&1&0&0\\
    0&1&1&0&1\\
    1&1&1&0&0\\
    0&0&0&1&0\\
    1&0&0&1&0\\
    0&1&0&1&1\\
    1&1&0&1&0\\
    0&0&1&1&0\\
    1&0&1&1&0\\
    0&1&1&1&1\\
    1&1&1&1&1\\
  	\hline
	\end{tabular}
	\caption{Truth Table of $f$}
\end{table}
\par The Hamming weight of $f$ is the number of 1s that $f$ has when evaluated
at every point in the $\gftwo^n$. Formally, this can be written
\[
wt(f)=\lvert\{u\in\gftwo^n:f(u)=1\}\rvert.
\]
\par A truth table is not a very efficient method to define a Boolean function.
An algebraic definition of $f$ is desirable. By writing $f$ algebraically, more
properties of $f$ will become apparent. As a first step toward defining $f$
algebraically a function will be one-to-one and onto function will be defined
which maps every $f$ in $\mathcal{BF}_n$ to a vector in $\gftwo^{2^n}$. This
will be the function $V:\mathcal{BF}_n\rightarrow\gftwo^{2^n}$ such that
\begin{equation}\label{eqn:bool-vector}
V(f(u_0,\dots,u_{n-1})):=(f(u_0),\dots,f(u_{n-1})).
\end{equation}
It is trivial to show that addition is homomorphic under $V$, in other words
\[
V(f_1\oplus f_2)=V(f_1)\oplus V(f_2).
\]
Then a basis of $\gftwo^{2n}$ must relate to an equivalent basis of
$\mathcal{BF}_n$. Take the standard basis vectors of $\gftwo^{2^n}$:
\begin{align*}
	e_0&=(1,0,\dots)\\
	e_1&=(0,1,\dots)\\
	\vdots \\
	e_{2^n-1}&=(0,\dots,0,1).
\end{align*}

\par Every vector of $\gftwo^{2^n}$ can be written as a linear combination of these
basis vectors. Equivalently, every Boolean function is a linear combination of the
inverse maps of the basis vectors under $V$. Let $u\in\gftwo^{2^n}$ and define
$f=V^{-1}(u)$ and $f_i=V^{-1}(e_i)$. Then there exist a set of $c_i\in\gftwo$ such that
\begin{align*}
						  u        &=c_0e_0\oplus\cdots\oplus c_{2^n-1}e_{2^n-1} \\
	\Rightarrow V^{-1}(u)&=c_0V^{-1}(e_0)\oplus\cdots\oplus c_{2^n-1}V^{-1}(e_{2^n-1}) \\
	\Rightarrow f        &=c_0f_0\oplus\cdots\oplus c_{2^n-1}f_{2^n-1}.\\
\end{align*}

\par The Boolean fucntions $f_i=V^{-1}(e_i)$ are alled {\em atomic Boolean functions}
because they are the building blocks of every Boolean function. Now the function
$f$ defined in Figure \ref{fig:truth-table} can be written as a linear combination
of the atomic Boolean functions in $\mathcal{BF}_4$. Since the coefficients in the
linear combinations are either 0 or 1, it is true that every Boolean function can be
written as the sum of $wt(f)$ atomic Boolean functions.
\begin{table}[h!]\label{tab:atomic-f}
	\centering
  \begin{tabular}{|c|c|c|c|c|c|c|c|c|c|c|c|}
    \hline
    $x_0$&$x_1$&$x_2$&$x_3$&$f$&$f_1$&$f_2$&$f_4$&$f_6$&$f_{10}$&$f_{14}$&$f_{15}$\\
    \hline
    0&0&0&0&0&0&0&0&0&0&0&0\\
    1&0&0&0&1&1&0&0&0&0&0&0\\
    0&1&0&0&1&0&1&0&0&0&0&0\\
    1&1&0&0&0&0&0&0&0&0&0&0\\
    0&0&1&0&1&0&0&1&0&0&0&0\\
    1&0&1&0&0&0&0&0&0&0&0&0\\
    0&1&1&0&1&0&0&0&1&0&0&0\\
    1&1&1&0&0&0&0&0&0&0&0&0\\
    0&0&0&1&0&0&0&0&0&0&0&0\\
    1&0&0&1&0&0&0&0&0&0&0&0\\
    0&1&0&1&1&0&0&0&0&1&0&0\\
    1&1&0&1&0&0&0&0&0&0&0&0\\
    0&0&1&1&0&0&0&0&0&0&0&0\\
    1&0&1&1&0&0&0&0&0&0&0&0\\
    0&1&1&1&1&0&0&0&0&0&1&0\\
    1&1&1&1&1&0&0&0&0&0&0&1\\
  	\hline
	\end{tabular}
	\caption{$f$ broken into atomic Boolean function in $\mathcal{BF}_4$}
\end{table}
\par From Figure \ref{fig:atomic-f} the following equation should be clear:
\[
f=f_1+f_2+f_4+f_6+f_{10}+f_{14}+f_{15}.
\]

\par So far, there has been no discussion about the $n$-variable polynomical
representation of a Boolean function. It however should be clear now that
if we can write the atomic Boolean functions as polynomials, then the
rest of the Boolean functions should follow. Though, it is not immediately
obvious how to write the each atomic Boolean function as a polynomial in
$\gftwo[x_0,\cdots,x_{n-1}]/\allowbreak(x_0^2\oplus\allowbreak x_1,\cdots,x_{n-1}^2\oplus\allowbreak x_{n-1})$.

\par First, observe that $wt(f_i)$ always equals 1. Thus
\begin{align*}
	f_i(u)&=1\hspace{5mm}u=B^{-1}(i) \\
			  &=0\hspace{5mm}{\rm otherwise.}
\end{align*}
It follows that
\begin{equation}\label{eqn:atomic-ANF}
	f_i=\bigg(\prod_{j\in supp(f_i)}x_j\bigg)\bigg(\prod_{j\not\in supp(f_i)}(1\oplus x_j)\bigg).
\end{equation}
Equation \ref{eqn:atomic-ANF} equals 1 at the vector $B^{-1}(i)$ and zero
everywhere else. Continuing the example with the the function $f$ will make
this more apparent. The atomic functions which $f$ is the sum of are written
as polynomials below:

\begin{align*}
  f_1   &=(1\oplus x_3)(1\oplus x_2)(1\oplus x_1)x_0\\
        &=x_0 \oplus x_1x_0 \oplus x_2x_0 \oplus x_2x_1x_0 \oplus x_3x_0 \oplus x_3x_1x_0
        \oplus x_3x_2x_0 \oplus x_3x_2x_1x_0\\
  f_2   &=(1\oplus x_3)(1\oplus x_2)x_1(1\oplus x_0)\\
  f_4   &=(1\oplus x_3)x_2(1\oplus x_1)(1\oplus x_0)\\
  f_6   &=(1\oplus x_3)x_2x_1(1\oplus x_0)\\
  f_{10}&=x_3(1\oplus x_2)x_1(1\oplus x_0)\\
  f_{14}&=x_3x_2x_1(1\oplus x_0)\\
  f_{15}&=x_3x_2x_1x_0\\
\end{align*}

\par Therefore,
\[
	f=x_0x_1x_2x_3\oplus x_0x_1x_3 \oplus x_0x_3 \oplus x_0 \oplus x_1x_2x_3 \oplus
	x_1x_2\oplus x_1\oplus x_2x_3 \oplus x_2.
\]

\par All of the $n$-variable atomic Boolean functions are linearly independent and
act as a basis for $\mathcal{BF}_n$. It follows that each $n$-variable Boolean
function is uniquely represeneted as a polynomial with coefficients in $\gftwo$:
\[
f(x)=\sum_{I\in\mathcal{P}(N)}a_I\bigg(\prod_{i\in I}x_i\bigg).
\]
