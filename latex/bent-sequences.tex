\section{Bent Sequences}
\par Sequences generated using bent functions have nice cryptographic properties 
because of their perfect nonlinearity. These sequences can be generated multiple
ways. Two easy examples are a filtering function on a shift register producing
an $m$-sequence or a shift register which uses $n$ different shift registers as
input into a bent function. These two techniques are discussed by Carlet \cite{col:c06}.
Both of constructions use input vectors from $gftwo^n$ in a psuedorandom order
to generate the sequence. Before scrambling the input in this way, the sequences
generated by lexicographic ordering of input vectors is considered.

\par Using the lexicographic ordering, the finite sequence generated will be the column of
the outputs for the Boolean function read from the truth table of the Boolean fucntion.
For example, the {\em finite Boolean\ sequence} using a lexicographic on the Boolean function
$f$ in \ref{tab:truth-table} is
\[
(0,1,1,0,1,0,1,0,0,0,1,0,0,0,1,1).
\]

\par 
