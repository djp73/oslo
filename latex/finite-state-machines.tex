\section{Shift Registers}
\par The {\it feedback with carry shift register} (or FCSR) is a type of
shift register used in stream ciphers. Though there exist attacks against
this type of shift register by itself, it is possible to combine FCSRs
together in ways that no known attacks are useful. This however does not
guarantee security because the combining of the FCSRs greatly increases the
complexity of the cryptanalysis on the stream cipher. Despite the security
challenges, the speed of an FCSR implementation is very attractive for
engineers of hardware based cryptosystems.

\par This section will be used as an introduction to finite state machines
and feedback with carry shift registers. The FCSR will be considered in the
binary case and analyzed using $\zzz_2$. This analysis is then extended to
the case when FCSRs produce bent sequences, the theme of Section
\ref{sec:Boolean Sequences}

\subsection{Finite State Machines}
\par It is appropriate to preface the discussion about FCSRs with general finite
state machines. Solomon W. Golomb's book {\em Shift Register Sequences}
\cite{bk:g82}, written in 1967 and revised in 1982, established a definition of
finite state machines and shift
registers used in much of the literature today.

%This section will begin by
%defining the finite state machine according to Golomb and move toward the
%definition of generalized shift registers. In the next section, FCSRs will
%be defined with Golomb's style in mind.

\begin{definition}\label{finite-state-machine}
  A {\em finite state machine} consists of a finite collection of {\em states}
  $K$, which sequentially accepts a sequence of {\em inputs} from a finite set
  $A$, and produces a sequence of {\em outputs} from a finite set
  $B$. Moreover, there is an {\em output function} $\mu$ which computes
  the present output as a fixed function of present input and present state,
  and a
  {\em next state function} $\delta$ which computes the next states as a fixed
  function of present input and present state. In a more mathematical manner,
  $\mu$ and $\delta$ are defined such that
  \begin{eqnarray}
    \mu:K \times A \rightarrow B \quad &\mu(k_n,a_n)=b_n \\
    \delta:K \times A \rightarrow K \quad &\delta(k_n,a_n)=k_{n+1}
  \end{eqnarray}
\end{definition}

% insert table example of a finite state machine

\par The most fundamental observation by Golomb is the following theorem.

\begin{theorem}\label{thm:golomb-2}
  If the input sequence to a finite state machine is eventually periodic, then
  the output sequence is eventually periodic.
\end{theorem}
\begin{proof}
  Let $p$ be the period of the inputs once the machine becomes periodic at time
  $t$. Then, for $h>0$ and $c>t$, $a_c=a_{c+hp}$. Since $K$ is finite, there
  must be $r>s>t$ such that, for some $h>0$ such that,
  \[
  k_{r+1}=\delta(k_r,a_r)=\delta(k_s,a_{r+hp})=k_{s+1}.
  \]
  It should also be clear that $a_{r+i}=a_{r+i+hp}$ for $h>0$. So by induction,
  $\forall i>0$
  \[
  k_{r+i+1}=\delta(k_{r+i},a_{r+i})=\delta(k_{s+i},a_{r+i+hp})=k_{s+i+1}
  \]
  Finally, this proves $b_{r+i+1}=b_{s+i+1}$. Thus, the eventual period of this
  machine is $r-s$.
\end{proof}

\par The next object defined is called an $N$-ary $n$-stage machine. It can be
used to represent any finite state machine. It is also a natural generalization
of shift registers, so thinking of finite state machines in the context of
$N$-ary $n$-state machines will make the transition to talking about shift
registers much smoother.

\begin{definition}\label{N-ary-n-stage-machine}
  Choose $n,m,r\in\nnn$. Then define a finite state machine with the
  following sets:
  \begin{enumerate}[1.]
    \item $D=\{0,\dots,N-1\}$. This set contains the {\em $N$-ary digits} of the
      machine.
    \item $K=\{\sum_{i=0}^{n}x_iN^i:x_i\in D\}$. This set contains the
      {\em $N$-ary states} of the machine.
    \item $A=\{\sum_{i=0}^{m}y_iN^i:y_i\in D\}$. This set contains the
      {\em $N$-ary inputs} of the machine.
    \item $B=\{\sum_{i=0}^{r}z_iN^i:z_i\in D\}$. This set contains the
      {\em $N$-ary outputs} of the machine.
    \item $F=\{f_i(x_0,\dots,x_n,y_0,\dots,y_m):0\le i<n\}$. This set contains
      the {\em $N$-ary next state functions} of the machine.
    \item $G=\{g_i(x_0,\dots,x_n,y_0,\dots,y_m):0\le i<r\}$. This set contains
      the {\em $N$-ary output functions} of the machine.
  \end{enumerate}
  The next state and output are determined from the current state and input by the
  following equations:
  \begin{eqnarray}
    x_i^*=f_i(x_0,\dots,x_{n-1},y_0,\dots,y_{m-1}) \quad 0\le i<n \\
    z_i=g_i(x_0,\dots,x_{n-1},y_0,\dots,y_{m-1}) \quad 0\le i<r
  \end{eqnarray}
  This finite state machine is called an {\em $N$-ary $n$-stage machine} and
  will be denoted by $\mathcal{M}(N,n,m,r)$.
\end{definition}

\par By making the state the input to the machine as well, this machine
becomes autonomous in the sense that it no longer needs outside input. Then
each new state and output is based on the previous state of the machine. For
$N=2$, $f_i$ and $g_i$ are Boolean functions on $n+m$ variables.
A binary $n$-stage machine can be defined by $n+r$ Boolean functions each
on $n+m$ variables.

\subsection{Feedback with Carry Shift Registers}

\par In the set of autonomous finite state machine is a type of machine
called a {\it shift register}. The variables making up the state of a shift
register pass their values directly to the next variable in the state until
the value is pushed out of the register as the output.

\begin{definition}\label{def:shift-register}
  Let $\mathcal{M}(2,n,-1,0)$ be an binary $n$-stage machine with no input
  and exactly 1 output value. Also, let $g,f_i\in\BF_n$ where the states
  $(x_0,\cdots,x_{n-1})$ are the domain of $g$ and $f_i$, $f_i(x)=x_{i+1}$ for
  $0\leq i \leq n-2$, and $g(x)=x_0$. This type of machine will be called an {\it
  $n$-stage binary shift register} denoted by $\mathcal{SR}(2,n)$.
\end{definition}

\par When the function $f_{n-1}\in\BF_n$ is linear, then $\mathcal{SR}(2,n)$ is
called a {\it linear feedback shift register}. An LFSR is drawn in Figure
\ref{fig:LFSR}. This is the case where $f_{n-1}=\sum_{i=1}^nq_ix_{n-i}$ where
each $x_i,q_i \in \gftwo$. The $q_i$'s are called {\it taps}. In computer
science terms, to move forward in the sequence of states, each bit in the state
of the register shifts to the right one spot and then the newest bit enters
on the left end of the register and is the value given when each bit from
the previous state is AND'ed with its corresponding tap and then XOR'ed with
all the other AND'ed bit and taps.

\setlength{\unitlength}{1mm}
\begin{figure}[h!]
  \centering
  \begin{picture}(90,40)(-25,-25)
    %% State of the register
    \put(0,0){\framebox(10,10){$x_{n-1}$}}
    \put(10,0){\framebox(10,10){$x_{n-2}$}}
    \put(20,0){\framebox(10,10){$\dots$}}
    \put(30,0){\framebox(10,10){$x_{1}$}}
    \put(40,0){\framebox(10,10){$x_{0}$}}
    \put(50,5){\vector(4,0){10}}
    %% Taps
    \put(0,-13){\makebox(10,10){$q_1$}}
    \put(5,-8.5){\circle{9}}
    \put(10,-13){\makebox(10,10){$q_2$}}
    \put(15,-8.5){\circle{9}}
    \put(20,-13){\makebox(10,10){$\dots$}}
    \put(30,-13){\makebox(10,10){$q_{n-1}$}}
    \put(35,-8.5){\circle{9}}
    \put(40,-13){\makebox(10,10){$q_n$}}
    \put(45,-8.5){\circle{9}}
    %% Lines connecting Taps and the State of the Register
    \multiput(5,-0.2)(10,0){2}{\line(0,-6){4}}
    \multiput(35,-0.2)(10,0){2}{\line(0,-6){4}}
    %% Summer
    \put(-20,-25){\framebox(10,10){\Large $\sum$}}
    %% Lines from Taps to Summer
    \put(5,-12.8){\line(0,-6){3.5}}
    \put(15,-12.8){\line(0,-6){5.5}}
    \put(35,-12.8){\line(0,-6){8.5}}
    \put(45,-12.8){\line(0,-6){10.5}}
    \put(5,-16.3){\vector(-4,0){15}}
    \put(15,-18.3){\vector(-4,0){25}}
    \put(35,-21.3){\vector(-4,0){45}}
    \put(45,-23.3){\vector(-4,0){55}}
    %% Lines from Summer
    \put(-13,-15){\line(0,0){20}}
    \put(-13,5){\vector(4,0){13}}
    %% mod N
    \put(-13,5){\makebox(13,5){mod 2}}
  \end{picture}
  \caption{Linear Feedback Shift Register}
  \label{fig:LFSR}
\end{figure}

\par Linear feedback shift registers are well-studied in \cite{bk:g82}. By
using the Berlekamp-Massey algorithm it is possible to recover the state of
a given LFSR based on the output sequence. In fact given an LFSR output sequence
with period $2^n-1$, the Berlekamp-Massey algorithm will find a unique
minimal-length LFSR which generates this output after the first $2n$ digits have
been processed \cite{art:m69}. This algorithm is studied in more detail in
\cite{art:b06}.

\par In Figure \ref{fig:FCSR}, there is a memory cell attached to the linear
feedback shift register which adds some complexity to the register. In the
modified shift register shown in Figure \ref{fig:FCSR}, in each cycle, the
whole number quotient of $\sum_{i=1}^nq_ix_{n-i}$ is kept in the memory cell
$z$. The memory cell from the previous state of the modified shift register
is used to determined the sum modulo 2 for the newest bit in the state of
the register. A shift register modified in this way is known as a {\it
binary feedback with carry shift register}, or FCSR. For this paper, FCSRs
will only be considered in the binary case. Many of the theorems do
generalize the $N$-ary case, though sometimes it is necessar that $N$ be
prime.

\begin{figure}[h!]
  \centering
  \begin{picture}(90,40)(-40,-25)
    %% State of the register
    \put(0,0){\framebox(10,10){$x_{n-1}$}}
    \put(10,0){\framebox(10,10){$x_{n-2}$}}
    \put(20,0){\framebox(10,10){$\dots$}}
    \put(30,0){\framebox(10,10){$x_{1}$}}
    \put(40,0){\framebox(10,10){$x_{0}$}}
    \put(50,5){\vector(4,0){10}}
    %% Taps
    \put(0,-13){\makebox(10,10){$q_1$}}
    \put(5,-8.5){\circle{9}}
    \put(10,-13){\makebox(10,10){$q_2$}}
    \put(15,-8.5){\circle{9}}
    \put(20,-13){\makebox(10,10){$\dots$}}
    \put(30,-13){\makebox(10,10){$q_{n-1}$}}
    \put(35,-8.5){\circle{9}}
    \put(40,-13){\makebox(10,10){$q_n$}}
    \put(45,-8.5){\circle{9}}
    %% Lines connecting Taps and the State of the Register
    \multiput(5,-0.2)(10,0){2}{\line(0,-6){4}}
    \multiput(35,-0.2)(10,0){2}{\line(0,-6){4}}
    %% Summer
    \put(-20,-25){\framebox(10,10){\Large $\sum$}}
    %% Lines from Taps to Summer
    \put(5,-12.8){\line(0,-6){3.5}}
    \put(15,-12.8){\line(0,-6){5.5}}
    \put(35,-12.8){\line(0,-6){8.5}}
    \put(45,-12.8){\line(0,-6){10.5}}
    \put(5,-16.3){\vector(-4,0){15}}
    \put(15,-18.3){\vector(-4,0){25}}
    \put(35,-21.3){\vector(-4,0){45}}
    \put(45,-23.3){\vector(-4,0){55}}
    %% Lines from Summer
    \put(-13,-15){\line(0,0){20}}
    \put(-13,5){\vector(4,0){13}}
    \put(-17,-15){\line(0,0){20}}
    \put(-17,5){\vector(-4,0){13}}
    %% Memory cell z
    \put(-40,0){\framebox(10,10){$z$}}
    %% Lines from memory cell to summer
    \put(-35,0){\line(0,-6){20}}
    \put(-35,-20){\vector(4,0){15}}
    %% div 2 mod 2
    \put(-13,5){\makebox(13,5){mod $2$}}
    \put(-29.5,5){\makebox(13,5){div $2$}}
  \end{picture}
  \caption{Binary Feedback with Carry Shift Register} 
  \label{fig:FCSR}
\end{figure}

\begin{definition}\label{afsr}
  Let $q_0,q_1,\dots,q_n\in\zzz/2$. A {\em binary feedback with carry shift register} 
  of length n with {\em taps} $q_1,\dots,q_n$ is a modified shift register
  whose states are collections
  \[
	(x_0,x_1,\dots,x_{n-1};z) \text{ where } x_i\in\gftwo \text{ and } z \in \zzz
  \]
  where $z$ is called the {\it memory cell}. The state changes according
  to the following rules:
  \begin{enumerate}[1.]
    \item Compute
      \[
      \sigma = \sum^n_{i=1}q_ia_{n-i}+z.
      \]
    \item The output is $x_0$.
    \item Then the new state $(x_0,x_1,\dots,x_{n-1};z)
      =(x_1,\dots,x_{n-1},\sigma\pmod2);\sigma(\text{div }2)$.
  \end{enumerate}
\end{definition}

\begin{lemma}
\par If the sequence $\alpha=(a_0,a_1,\cdots)$ where $a_i\in\{0,1\}$ and
$a_i\equiv x_i\pmod2$, and $\zeta=(z_{n-1},z_n,z_{n+1},\cdots)$ where each
$z_i$ is the value of the memory cell for the corresponding $x_i$, then
these two sequences are related by the following linear recurrence
\begin{equation}\label{eqn:lin-rec}
  a_k+2z_k=q_1a_{k-1}+\cdots+q_na_{k-n}+z_{k-1} \text{ for } k\geq n.
\end{equation}
\end{lemma}
