\section{Finite State Machines}
\par It is interesting to consider FCSRs in the context of general finite state
machines. In 1967, Dr. Solomon W. Golomb published {\em Shift Register Sequences}
establishing a definition of finite state machines and shift registers used in
much of the literature today. This section will begin by defining the finite state
machine according to Golomb and move toward the definition of generalized shift
registers. In the next section, FCSRs will be defined with Golomb's style in mind.

\begin{definition}\label{finite-state-machine}
  A {\em finite state machine} consists of a finite collection of {\em states}
  $K$, sequentially accepts a sequence of {\em inputs} from a finite set
  $A$, and produces a sequence of {\em outputs} from a finite set
  $B$. Moreover, there is an {\em output function} $\mu$ which computes
  the present output as a fixed function of present input and present state, and a
  {\em next state function} $\delta$ which computes the next states as a fixed
  function of present input and present state. In a more mathematical manner,
  $\mu$ and $\delta$ are defined such that
  \begin{eqnarray}
    \mu:K \times A \rightarrow B \quad &\mu(k_n,a_n)=b_n \\
    \delta:K \times A \rightarrow K \quad &\delta(k_n,a_n)=k_{n+1}
  \end{eqnarray}
\end{definition}

% insert table example of a finite state machine

\par Golomb presents two important theorems about these machines. Both deal with
the periodicity of any finite state machine, which include FCSRs. The theorems
and proofs are presented here.

%\begin{theorem}\label{thm:golomb-1}
%  If the input to a finite state machine is eventually constant, then the output
%  is eventually periodic.
%\end{theorem}
%\begin{proof}
%  Let $t$ be the time when the input becomes constant, so $a_t=a_{t+1}=\dots$.
%  Because $K$ is a finite collection of states, there exists times $r>s>t$ such
%  that $k_r=k_s$. Then, by induction, $\forall i>0$,
%  \[
%  k_{r+i+1}=\delta(k_{r+i},a_{r+i})=\delta(k_{s+i},a_{s+i})=k_{s+i+1}
%  \]
%  Therefore,
%  \[
%  b_{r+i+1}=\mu(k_{r+i+1},a_{r+i+1})=\delta(k_{s+i+1},a_{s+i+1})=b_{s+i+1}
%  \]
%  Thus, the eventual period of this machine is $r-s$.
%\end{proof}

\begin{theorem}\label{thm:golomb-2}
  If the input sequence to a finite state machine is eventually periodic, then the
  output sequence is eventually periodic.
\end{theorem}
\begin{proof}
  Let $p$ be the period of the inputs once the machine becomes periodic at time $t$.
  Then, for $h>0$ and $c>t$, $a_c=a_{c+hp}$. Similar to the proof of Theorem
  \ref{thm:golomb-1}, using the fact that $K$ is finite, there must be
  $r>s>t$ such that, for some $h>0$,
  \[
  k_{r+1}=\delta(k_r,a_r)=\delta(k_s,a_{r+hp})=k_{s+1}.
  \]
  It should also be clear that $a_{r+i}=a_{r+i+hp}$ for $h>0$. So by induction,
  $\forall i>0$
  \[
  k_{r+i+1}=\delta(k_{r+i},a_{r+i})=\delta(k_{s+i},a_{r+i+hp})=k_{s+i+1}
  \]
  Finally, this proves $b_{r+i+1}=b_{s+i+1}$. Thus, the eventual period of this
  machine is $r-s$.
\end{proof}

\par The next object defined is called an $N$-ary $n$-stage machine. It can be used
to represent any finite state machine. It is also a natural generalization of
shift registers, so thinking of finite state machines in the context of $N$-ary
$n$-state machines will make the transition to talking about shift registers much
smoother.

\begin{definition}\label{N-ary-n-stage-machine}
  Choose $n,m,r\in\nnn$. An {\em $N$-ary $n$-stage machine} consists of the
  following:
  \begin{enumerate}[1.]
    \item $D=\{0,\dots,N-1\}$. This set contains the {\em $N$-ary digits} of the
      machine.
    \item $K=\{\sum_{i=0}^{n}x_iN^i:x_i\in D\}$. This set contains the
      {\em $N$-ary states} of the machine.
    \item $A=\{\sum_{i=0}^{m}y_iN^i:y_i\in D\}$. This set contains the
      {\em $N$-ary inputs} of the machine.
    \item $B=\{\sum_{i=0}^{r}z_iN^i:z_i\in D\}$. This set contains the
      {\em $N$-ary outputs} of the machine.
    \item $F=\{f_i(x_0,\dots,x_n,y_0,\dots,y_m):0\le i<n\}$. This set contains
      the {\em $N$-ary next state functions} of the machine.
    \item $G=\{g_i(x_0,\dots,x_n,y_0,\dots,y_m):0\le i<r\}$. This set contains
      the {\em $N$-ary output functions} of the machine.
  \end{enumerate}
  The next state and output are determined from the current state and input by the
  following equations:
  \begin{eqnarray}
    x_i^*=f_i(x_0,\dots,x_n,y_0,\dots,y_m) \quad 0\le i<n \\
    z_i=g_i(x_0,\dots,x_n,y_0,\dots,y_m) \quad 0\le i<r
  \end{eqnarray}
\end{definition}
