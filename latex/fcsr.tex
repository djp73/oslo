\section{Feedback with Carry Shift Registers}
\par First, the definition of a {\em finite state machine}
according to Golomb's famous book {\em Shift Register Sequences}
is given.

\begin{definition}\label{finite-state-machine}
  A {\em finite state machine} consists of a finite collection of {\em states}
  $K$, sequentially accepts a sequence of {\em inputs} from a finite set
  $A$, and produces a sequence of {\em outputs} from a finite set
  $B$. Moreover, there is an {\em output function} $\mu$ which computes
  the present output as a fixed function of present input and present state, and a
  {\em next state function} $\delta$ which computes the next states as a fixed
  function of present input and present state. In a more mathematical manner,
  $\mu$ and $\delta$ are defined such that
  \begin{eqnarray}
    \mu:K \times A \rightarrow B \quad &\mu(k_n,a_n)=b_n \\
    \delta:K \times A \rightarrow K \quad &\delta(k_n,a_n)=k_{n+1}
  \end{eqnarray}
\end{definition}

\par There are two important theorems Golomb presents as his first theorems about
these machines. The theorems and proofs are presented here.

\begin{theorem}\label{thm:golomb-1}
  If the input to a finite state machine is eventually constant, then the output
  is eventually periodic.
\end{theorem}
\begin{proof}
  Let $t$ be the time when the input becomes constant, so $a_t=a_{t+1}=\dots$.
  Because $K$ is a finite collection of states, there exists times $r>s>t$ such
  that $k_r=k_s$. Then, by induction, $\forall i>0$,
  \[
  k_{r+i+1}=\delta(k_{r+i},a_{r+i})=\delta(k_{s+i},a_{s+i})=k_{s+i+1}
  \]
  Therefore,
  \[
  b_{r+i+1}=\mu(k_{r+i+1},a_{r+i+1})=\delta(k_{s+i+1},a_{s+i+1})=b_{s+i+1}
  \]
  Thus, the eventual period of this machine is $r-s$.
\end{proof}

\begin{theorem}\label{thm:golomb-2}
  If the input sequence to a finite state machine is eventually periodic, then the
  output sequence is eventually periodic.
\end{theorem}
\begin{proof}
  Let $p$ be the period of the inputs once the machine becomes periodic at time $t$.
  Then, for $h>0$ and $c>t$, $a_c=a_{c+hp}$. Similar to the proof of Theorem
  \ref{thm:golomb-1}, using the fact that $K$ is finite, there must be
  $r>s>t$ such that, for some $h>0$,
  \[
  k_{r+1}=\delta(k_r,a_r)=\delta(k_s,a_{r+hp})=k_{s+1}.
  \]
  It should also be clear that $a_{r+i}=a_{r+i+hp}$ for $h>0$. So by induction,
  $\forall i>0$
  \[
  k_{r+i+1}=\delta(k_{r+i},a_{r+i})=\delta(k_{s+i},a_{r+i+hp})=k_{s+i+1}
  \]
  Finally, this proves $b_{r+i+1}=b_{s+i+1}$. Thus, the eventual period of this
  machine is $r-s$.
\end{proof}

\par A {\em feedback with carry shift register} is a feedback shift register
which uses a linear combination each state toA description of $N$-ary feedback with
carry shift registers is defined here. The definition of the register follows the
one given in Andrew Klapper's book.

\begin{definition}\label{afsr}
  Let $q_0,q_1,\dots,q_m\in\zzz/(p)$ for $p\in\zzz$ and assume that
  $q_0\not\equiv0\pmod p$.
  An {\em algebraic feedback shift register} (or {\em AFSR}) over ($\zzz$,$p$,$S$)
  of length m with {\em multipliers} or {\em taps} $q_0,q_1,\dots,q_m$ is a discrete
  state machine whose states are  collections
  \[
  (a_0,a_1,\dots,a_{m-1};z) \ where \ a_i\in S \ and \ z \in \zzz
  \]
  consisting of cell contents $a_i$ and memory $z$. The state changes according to
  the following rules:
  \begin{enumerate}[1.]
    \item Compute
      \[
      \sigma = \sum^m_{i=1}q_ia_{m-i}+z.
      \]
    \item Find $a_m\in S$ such that $-q_0a_m\equiv\sigma\pmod p$. That is
      $a_m\equiv-q_{0}^{-1}\sigma\pmod p$.
    \item Replace $(a_0,\dots,a_{m-1})$ by $(a_1,\dots,a_m)$ and replace $z$ by
      $\sigma({\rm div} p) = (\sigma+q_0a_m)/p$.
  \end{enumerate}
\end{definition}
