\section{Feedback with Carry Shift Registers}

\par A {\em feedback with carry shift register} is a feedback shift register
which uses a linear combination each state toA description of $N$-ary feedback with
carry shift registers is defined here. The definition of the register follows the
one given in Andrew Klapper's book \cite{bk:gk12}.

\begin{definition}\label{afsr}
  Let $q_0,q_1,\dots,q_m\in\zzz/(p)$ for $p\in\zzz$ and assume that
  $q_0\not\equiv0\pmod p$.
  An {\em algebraic feedback shift register} (or {\em AFSR}) over ($\zzz$,$p$,$S$)
  of length m with {\em multipliers} or {\em taps} $q_0,q_1,\dots,q_m$ is a discrete
  state machine whose states are  collections
  \[
	(a_0,a_1,\dots,a_{m-1};z) \ {\rm where} \ a_i\in S \ {\rm and} \ z \in \zzz
  \]
  consisting of cell contents $a_i$ and memory $z$. The state changes according to
  the following rules:
  \begin{enumerate}[1.]
    \item Compute
      \[
      \sigma = \sum^m_{i=1}q_ia_{m-i}+z.
      \]
    \item Find $a_m\in S$ such that $-q_0a_m\equiv\sigma\pmod p$. That is
      $a_m\equiv-q_{0}^{-1}\sigma\pmod p$.
    \item Replace $(a_0,\dots,a_{m-1})$ by $(a_1,\dots,a_m)$ and replace $z$ by
      $\sigma({\rm div} p) = (\sigma+q_0a_m)/p$.
  \end{enumerate}
\end{definition}
