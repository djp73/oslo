\section{Boolean Sequences}
\par The interest of this paper is stream ciphers, and there are a few
different ways to use bent functions in the implementation of a stream
cipher. Sequences generated using bent functions have nice cryptographic
properties because of their perfect nonlinearity. These sequences can be
generated multiple ways. Two easy examples are a filtering function on a
shift register producing an $m$-sequence or a shift register which uses $n$
different shift registers as input into a bent function. These two
techniques are discussed by Carlet \cite{col:c06}. Both of these
constructions use input vectors from $\gftwo^n$ in a psuedorandom order to
generate the sequence. Before scrambling the input in this way, the
sequences generated by lexicographic ordering of input vectors is
considered.

\begin{definition}
  Let $(a_n)$ be a sequence. If $T$ is the smallest integer such that
  $a_i=a_{i+T}$, then the {\em minimal\ period} of $(a_n)$ is $T$.
\end{definition}

\begin{definition}\label{def:lex-Bool-seq}
  Let $f\in\BF_n$ and $v_i\in\gftwo^n$ such that $v_i=B^{-1}(i)$ for
  $0\leq i<2^n$. Then,
  \begin{equation}
    seq(f)=(f(v_0),f(v_1),\cdots,f(v_{2^n-1}),f(v_0),\cdots)
  \end{equation}
  is a {\em lexicographical\ Boolean\ sequence}.
\end{definition}

\par Defined in this way, all lexicographical Boolean sequences have a
minimal period at most $2^n$. Using the lexicographic ordering, the Boolean
sequence generated will be repeated columns of the outputs for the Boolean
function read from the truth table of the Boolean fucntion. For example, the
lexicographic Boolean sequence of $f$ in Table \ref{tab:truth-table} is
\[
(0,1,1,0,1,0,1,0,0,0,1,0,0,0,1,1,0,1,1,0,1,0,1,0,0,0,1,0,0,0,1,1,\cdots).
\]

\begin{theorem}
  The lexicographical Boolean sequence of a Bent function has a period
  exactly $2^n$.
\end{theorem}
\begin{proof}
  For $x_i,\lambda_i\in\gftwo$ and $0\leq i\leq n-2$, define
  $(x,1)=\allowbreak(x_0,\cdots,\allowbreak x_{n-2},1)$,
  $(x,0)=\allowbreak(x_0,\cdots,\allowbreak x_{n-2},0)$, and
  $(\lambda,1)=\allowbreak(\lambda_0,\cdots,\allowbreak \lambda_{n-2},1)$.
  Suppose $f\in\BF_n$ and $seq(f)$ has a period $T=2^j<2^n$. Then,
  $f(x,0)=f(x,1)$.
  \begin{align*}
    c(\lambda,1)&=\frac{1}{2^{n/2}}
      \left(\sum_{x\in\gftwo^{n-1}}
        {(-1)^{f(x,0)+(x,0)\cdot(\lambda,1)}+
        (-1)^{f(x,1)+(x,1)\cdot(\lambda,1)}}
      \right)\\
    &=\frac{1}{2^{n/2}}
      \left(\sum_{x\in\gftwo^{n-1}}
        {(-1)^{f(x,0)}\left((-1)^{(x,0)\cdot(\lambda,1)}+
        (-1)^{(x,1)\cdot(\lambda,1)}\right)}
      \right)\\
    &=\frac{1}{2^{n/2}}
      \left(\sum_{x\in\gftwo^{n-1}}
      {(-1)^{f(x,0)}\left((-1)^{0\cdot1}+(-1)^{1\cdot1}\right)}
      \right)\\
    &=\frac{1}{2^{n/2}}
      \left(\sum_{x\in\gftwo^{n-1}}{(-1)^{f(x,0)}\cdot0}\right)\\
    &=0.
  \end{align*}
  One of the Fourier coefficients of $f$ must equal zero. Thus, $f$ cannot
  be a Bent function. Clearly, every lexicographical Boolean sequence has a
  minimal period at most $2^n$. Therefore, if $g$ is a Bent function, then
  $seq(g)$ has period exactly $2^n$.
\end{proof}

\par Boolean sequences will be considered as 2-adic exansions of rational
numbers.

\begin{definition}\label{2-adic-ex}
  Let $f\in\BF_n$ and $v_i\in\gftwo^n$ such that $v_i=B^{-1}(i)$ for
  $0\leq i<2^n$. Then,
  \begin{equation}
    \alpha_f=(f(v_0),f(v_0)+f(v_1)\cdot2,\cdots,\allowbreak
      f(v_0)+\cdots\allowbreak+f(v_i)\cdot2^i,\allowbreak\cdots)
  \end{equation}
  where $\alpha_f\in\zzz_2$ is called the {\em 2-adic\ expansion} of $f$.
\end{definition}

\begin{lemma}
  The digit representation of $\alpha_f$ is $seq(f)$.
\end{lemma}

\par Recall the Maiorana-McFarland class of Boolean functions from Section
\ref{subsec:bent-constructions}, and consider
the subset of these functions where $g(y)=0$. Then the following theorem is
true.

\begin{theorem}
  $\log_2(\alpha_{x\cdot\pi(y)})=2^{n/2}+2^{\bar{\pi}(y_0)}$
\end{theorem}
