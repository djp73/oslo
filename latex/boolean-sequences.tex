\section{Boolean Sequences}
\par The interest of this paper is stream ciphers, and there are a few
different ways to use bent functions in the implementation of a stream
cipher.

\begin{definition}\label{def:lex-Bool-seq}
  Let $f\in\BF_n$ and $v_i\in\gftwo^n$ such that $v_i=B^{-1}(i)$ for
  $0\leq i<2^n$. Then,
  \begin{equation}
    seq(f)=(f(v_0),f(v_1),\cdots,f(v_{2^n-1}),f(v_0),\cdots)
  \end{equation}
  is a {\em lexicographical\ Boolean\ sequence}.
\end{definition}

\begin{lemma}\label{lem:Bseq-period}
The lexicographical Boolean sequence has a maximal period of $2^n$.
\end{lemma}

\begin{theorem}
  The lexicographical Boolean sequence of a Bent function has a period
  exactly $2^n$.
\end{theorem}
\begin{proof}
  Let $(x,1)\in\gftwo^n$ so that $(x,1)=(x_0,\cdots,x_{n-2},1)$. Define
  $(x,0)$ and $(\lambda,1)$ in a similar way. Suppose $f\in\BF_n$ and
  $seq(f)$ has a period $T=2^j<2^n$. Then, $f(x,0)=f(x,1)$.
  \begin{align*}
    c(\lambda,1)&=\frac{1}{2^{n/2}}
      \left(\sum_{x\in\gftwo^{n-1}}
        {(-1)^{f(x,0)+(x,0)\cdot(\lambda,1)}+
        (-1)^{f(x,1)+(x,1)\cdot(\lambda,1)}}
      \right)\\
    &=\frac{1}{2^{n/2}}
      \left(\sum_{x\in\gftwo^{n-1}}
        {(-1)^{f(x,0)}\left((-1)^{(x,0)\cdot(\lambda,1)}+
        (-1)^{(x,1)\cdot(\lambda,1)}\right)}
      \right)\\
    &=\frac{1}{2^{n/2}}
      \left(\sum_{x\in\gftwo^{n-1}}
      {(-1)^{f(x,0)}\left((-1)^{0\cdot1}+(-1)^{1\cdot1}\right)}
      \right)\\
    &=\frac{1}{2^{n/2}}
      \left(\sum_{x\in\gftwo^{n-1}}{(-1)^{f(x,0)}\cdot0}\right)\\
    &=0.
  \end{align*}
  One of the Fourier coefficients of $f$ must equal zero. Thus, $f$ cannot
  be a Bent function. By Lemma\ \ref{lem:Bseq-period},\ every
  lexicographical Boolean sequence has a maximal period of $2^n$. Therefore,
  if $g$ is a Bent function, then $seq(g)$ has period exactly $2^n$.
\end{proof}

\par Boolean sequences will be considered as the coefficients of the power
series used to define 2-adic integers.

\begin{definition}\label{2-adic-ex}
  Let $f\in\BF_n$ and $v_i\in\gftwo^n$ such that $v_i=B^{-1}(i)$ for
  $0\leq i<2^n$. Then,
  \begin{equation}
    \alpha_f=(f(v_0),f(v_0)+f(v_1)\cdot2,\cdots,\allowbreak
      f(v_0)+\cdots\allowbreak+f(v_i)\cdot2^i,\allowbreak\cdots)
  \end{equation}
  where $\alpha_f\in\zzz_2$ is called the {\em 2-adic\ expansion} of $f$.
\end{definition}

\begin{lemma}
  The digit representation of $\alpha_f$ is $seq(f)$.
\end{lemma}

\par Recall the Maiorana-McFarland class of Boolean functions, and consider
the subset of these functions where $g(y)=0$. Then the following theorem is
true.

\begin{theorem}
  $\log_2(\alpha_{x\cdot\pi(y)})=2^{n/2}+2^{\bar{\pi}(y_0)}$
\end{theorem}
